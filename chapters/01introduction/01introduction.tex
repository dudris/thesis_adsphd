% !TeX root = ../../thesis.tex
\chapter{Introduction} \label{ch_introduction}

\newglossaryentry{md}{name={MD},description={molecular dynamics}}



\section{General introduction}
Crystallographic texture has a marked influence on the properties of polycrystalline coatings. Seemingly unrelated properties such as corrosion resistance, hardness, magnetic properties, porosity, and contact resistance are all texture dependent~\cite{Schlesinger2010}. Some properties such as macroscopic magnetic anisotropy do not even exist if the sample is not textured~\cite{Dinnebier2008}.

Every deposition method has its specific parameters used to control the process. These are used to optimize the resulting textures, but the relation to the fundamental texture-forming quantities (i.e. the local conditions on the deposited material surface) is usually missing. The problem is that they are inaccessible after the deposition and it is not clear how they could be measured during the deposition. 

Irrespective of which type of deposition technique is used, texture formation is controlled by interfacial energy, surface diffusivity of adatoms, grain boundary mobility and grain boundary and lattice diffusion~\cite{Szpunar1997, Suwas2014}. These properties together then affect the fundamental processes: nucleation, crystal growth and grain boundary motion~\cite{Barna1998}. Understanding how the texture in films is formed and how it can be controlled during the deposition process, is beneficial for the development of methods of texture optimization~\cite{Szpunar1997}. 

The anisotropy in crystal growth in general must originate from the anisotropy on the interfaces of the growing crystal with its surroundings - the substrate and the parent phase. In electrodeposition, the surface conditions may dynamically change due to local concentration evolution of the many present thermodynamic species in the solutions and their interactions with the deposit surface. 

In such situations, mathematical models of the process are useful to suggest some interpretation to the observed behaviors to gain better insight. This thesis elaborates on the possible impact of anisotropy in interface energy on the orientation selection during repeated nucleation on the course of the film growth. 

Of the deposition methods, the electrodeposition is particularly interesting, because
\begin{itemize}
	\item the (anisotropic) surface conditions are highly adjustable by additives and deposition conditions,
	\item the current or voltage transients provide valuable integral feedback on the surface processes in-situ, so there is more information available than in other methods,
	%	\item it can be carried out in a highly localized way (e.g. during electrochemical scanning microscopy), which allows higher control ???!,
	%	\item the direction of the deposition reaction is controlled by the sign of applied voltage, which allows pulsed regime, where deposition and stripping are repeated in rather short cycles. This adds completely new dynamics to the deposition process and provides additional perspective to tune the deposit properties, 
	\item it is industrially relevant.
\end{itemize}
For these reasons, this thesis focuses more on electrodeposition than on other deposition methods. However, the interface energy is relevant for all the others as well, hence the implications of its anisotropy are in principle applicable to any other deposition method. 

Electrodeposition is a highly industrially relevant process used for many applications like metal deposition for the fabrication of integrated circuits, deposition for magnetic recording devices (heads, disks), and deposition
of multilayer structures~\cite{Schlesinger2006}, to name a few. Similarly like in other methods, the electrodeposits are usually polycrystalline and exhibit pronounced crystallographic texture, which is affected by many factors. 

The interface energy is particularly important for the nucleation, which is in turn the main process of how grains with new orientations are introduced during the film growth. To the best of author's knowledge, the relation between the interface energy anisotropy and the anisotropy in nucleation barrier as function of local crystallographic orientation has not been fundamentally investigated. This work intends to bring more insight into this relation using different simulation approaches.




\section{Goals of the thesis}

\noindent\fbox{%
	\parbox{\textwidth}{%
		There are two primary goals of this thesis:
		\begin{enumerate}
			\item Develop a methodology based on the multi-phase-field model~\cite{Moelans2008} capable of determining the nucleation barrier anisotropy 
			\item Bring insight into the role of anisotropy in interface energy for nucleation during polycrystalline film growth and the possible impact on texture evolution.			
		\end{enumerate}
	}%
}


%% REFORMULATED USING CHAT GPT
%The primary aims of this thesis are twofold: first, to develop a methodology based on the multi-phase-field model proposed by Moelans et al. (2008) that is capable of determining nucleation barrier anisotropy; and second, to provide insights into the influence of anisotropy in interface energy on nucleation during polycrystalline film growth, with a focus on its potential effects on texture evolution.

To achieve these goals, several key tasks must be undertaken. The initial step involves developing and benchmarking the multi-phase-field model with the inclusion of anisotropic interface energy. Since existing variants of the model have not consistently incorporated inclination-dependent interface energy, it is necessary to derive the governing equations for each of these variants. Quantitative benchmark problems will then be defined to validate the model's implementation and assess its ability to capture the anisotropic curvature driving force. A custom numerical solver will be developed to facilitate these comparisons, and conclusions will be drawn regarding the most suitable model variant for the application at hand.

Another important task is to establish a methodology that uses the multi-phase-field model to determine nucleation barrier anisotropy. This involves conceptualizing the entire methodology, specifying the model requirements and simulation geometry, and, if needed, extending the model further. Once developed, the model will be implemented and validated by applying it to a problem with a known solution. Its applicability will then be demonstrated on a more complex problem where no known solution exists, showing the methodology’s practical value.

A subsequent critical step is investigating how the anisotropy of nucleation barrier relates to the anisotropy in interface energy and local crystallography. This requires a thorough analysis of the problem and the creation of a solver that can compute the nucleation barrier for each pair of substrate and particle orientations. The resulting solver will be used to explore how nucleation barriers depend on these orientations. In order to provide relevant resolution in the 2D space of orientations (of the nucleus and substrate), the solver must be reasonably efficient.

Furthermore, to illustrate how nucleation barrier anisotropy can influence texture evolution during film growth, a 2D Monte Carlo simulation will be developed. This simulation will incorporate an orientation-sampling algorithm based on anisotropic nucleation probability and will be validated to ensure accuracy. A parameter study will be conducted to examine the effect of isotropic and anisotropic interface energy on texture evolution under different conditions.

In addition to these technical milestones, the gained theoretical insights will be applied to a real-world practical system, demonstrating the broader relevance of the findings. Finally, the research will conclude with a synthesis of the results, outlining directions for future studies that can build upon the work presented in this thesis.

Overall, this work will contribute both methodologically and theoretically to the understanding of nucleation barrier anisotropy and its implications for polycrystalline film growth. 


\section{Thesis overview} \label{sec_Thesis_overview}
This thesis explores how anisotropy in surface energy affects the course of crystallographic texture evolution in polycrystalline coatings, with emphasis on orientation selection in repeated nucleation seen within the framework of classical nucleation theory. 

%In order to bring insight in this matter, the problem was first analyzed (KO1.1) \colorbox{red}{IN CHAPTER} and two different approaches to determine anisotropy in interface energy were developed (KO1.2 and KO1.3). 

The Chapters~\ref{ch_anisoIEintro} and~\ref{ch_introPF} provide the necessary fundamentals and relevant literature review. The subsequent chapters contain the original research carried out by the author.

%The Chapters~\ref{ch_paper1} and~\ref{ch_paper2} contain the results published as~\cite{Minar2022} and~\cite{Minar2024}, respectively. The Chapter~\ref{ch_NPA_PF_methodology} contains unpublished results on the phase-field methodology to obtain so-called shape factor of a heterogeneous nucleus. The shape factor can be related to the nucleation probability for the particular shape.

The Chapter~\ref{ch_anisoIEintro} provides fundamentals related to the surface energy, like the capillary vector formalism used to describe its anisotropy, interface stiffness, the Wulff shapes, the force balance in triple junctions and Winterbottom construction to obtain equilibrium stable shapes on a plane. Fundamentals of classical nucleation theory are provided as well.

The Chapter~\ref{ch_introPF} introduces the phase-field method. In a step-wise manner, the fundamentals of the method are covered. First, the Allen-Cahn equation is described, which represents phase evolution under the effect of isotropic curvature driving force. Then, the standard ways of adding a) anisotropy in interface energy, b) multiple phase-fields in the simulation domain (and both) are reviewed. Challenges and limitations of the various phase-field models are discussed. Also, a section was included, commenting on the importance of benchmarking the performance of applied multi-physics phase-field models and the current practice of their usage and presentation within the scientific community.

In the Chapter~\ref{ch_paper1}, there is a comprehensive description of an existing multi-phase field model and how and why it was extended. Three different variants of the same model were obtained. Two different benchmark problems to quantitatively test how the models reproduced the anisotropic curvature driving force were developed. A thorough parameter study was conducted to provide exhausting information on the model behavior in different settings. Only in a third benchmark problem involving triple junctions showed, that there was in fact a significant difference in the behavior of the model variants. Results in this chapter were published in~\cite{Minar2022}.

The Chapter~\ref{ch_NPA_PF_methodology} presents the \textit{domain scaling} methodology, which was developed to determine so-called shape factor of a heterogeneous nucleus. That quantity is closely tied with nucleation barrier and thus the nucleation probability. The multi-phase field model was used to obtain the equilibrium stable shape of nucleus needed for \textit{domain scaling} to work. Because the model extensions carried out in Chapter~\ref{ch_paper1} were still not sufficient to realize the developed methodology, additional extensions were introduced, building on top of the knowledge and implementation in Chapter~\ref{ch_paper1}. The methodology was validated and showcased on an example of a nucleus on top of a grain boundary within a substrate, which is a case where the shape does not have analytic solution.

Then follows the Chapter~\ref{ch_paper2}, which employs the so-called Winterbottom construction to obtain the equilibrium stable shape of the nucleus. Because this method is very computationally efficient, the nucleation barrier as function of the substrate and nucleus orientation could be determined with great resolution. The obtained \textit{shape-factor-orientation maps} are a visualization of the nucleation barrier orientation dependence (orientation of both the substrate and nucleus with anisotropic interface energy). These were then used as an input to a custom orientation-sampling algorithm for the nuclei. Using this, the orientation of nuclei introduced during a Monte Carlo simulation of 2D polycrystalline film growth was determined. A parametric study was conducted to identify the impact of the anisotropy in interface energy on the texture evolution, when compared to the isotropic interface energy. The obtained insights were used to qualitatively explain an abrupt change in texture which observed in electrodeposited nickel~\cite{Alimadadi2016}. Results in this chapter were published in~\cite{Minar2024}. 

The Chapter~\ref{ch_conclusion} summarizes the work and its main contributions and proposes directions for future research.

A number of appendices are included in the work to provide relevant information, which was not essential for understanding the results in the main text. They mostly contain mathematical derivations, details on implementation of numerical methods or other supplementary information improving transparency and reproducibility of this research. In order to achieve even higher reproducibility, all the source MATLAB code used to deliver the results in the published works was made available in online repositories~\cite{Minar2022dataset} and~\cite{Minar2023dataset}, together with usage instructions.




%Additionally, in the thesis were made relevant contribution to the scientific community developing and implementing mathematical models driven by anisotropic curvature driving force (e.g. multi-phase field method applied to grain growth) by providing two benchmark problems to measure physical accuracy of the models in this regard. At the same time, an existing multi-phase field model was extended to include inclination-dependent dependent interface energy via different combination of parameters than it was implemented before.
%
%The textures in as-deposited coatings are always affected by the deposition conditions, especially by the deposition method and further the temperature, deposition rate, materials of the deposit and substrate and other method-specific conditions. 


%
%A perspective approach for simulation of a growing film including nucleation is multi-phase field method, which allows inclusion of anisotropic interface energy, but also anisotropy of the kinetic coefficient (deposition rate).
%
%The goal 
%
%An established multi-phase field method was extended so that the equilibrium stable shapes could be obtained.



%This thesis explores the effect of anisotropic interface energy on nucleation and growth of polycrystalline deposits in various simulation concepts. A perspective approach for simulation of a growing film including nucleation is multi-phase field method, which also allows inclusion of anisotropic interface energy, but also anisotropy of the kinetic coefficient (deposition rate). There are multiple approaches to simulate growing polycrystal, ane of them being phase field. The different model formulations can have marked impact on the model behavior and limitations. However, with the current knowledge, it is not obvious how to choose the best model for any particular application, because there is no standard way to compare different models which should be representative of the same physics.
%
%Simulation of the growing polycrystal including nucleation at mesoscale is very challenging, because it occurs simultaneously at multiple length and time scales - the new grains originate at the atomistic scale (nucleation), but grow to mesoscopic dimensions under the deposition conditions. Capturing the local conditions necessary to approximate the atomistic processes requires resolution so fine that the mesoscopic simulation becomes unfeasible.
%
%Additionally, simulations of polycrystals are often computationally demanding, because in order to have sufficient grain statistics, large systems need to be taken into account.
%
%The interfaces introduce excess energy to an inhomogeneous thermodynamic system. By the excess energy minimization, the system is driven towards equilibrium and the resulting interfaces motion is due to the curvature driving force. This mathematical concept has been successfully applied to grain growth or motion of interfaces between soap bubbles. The anisotropic curvature driving force is also well defined
%






%%%%%%%%%%%%%%%%%%%%%%%%%%%%%%%%%%%%%%%%%%%%%%%%%%
% Keep the following \cleardoublepage at the end of this file, 
% otherwise \includeonly includes empty pages.
\cleardoublepage

% vim: tw=70 nocindent expandtab foldmethod=marker foldmarker={{{}{,}{}}}
