% !TeX root = ../../thesis.tex
\chapter{Abstract} \label{ch:abstract}
This study investigates how anisotropy in solid-liquid surface energy influences orientation selection during repeated nucleation in polycrystalline film growth and its impact on crystallographic texture formation. A novel methodology was developed to determine the nucleation probability as a function of substrate and nucleus orientation, based on the equilibrium shape of the nucleus under given conditions. Two approaches were devised to determine this shape: one based on a multi-phase field method and another on the analytical solution of the geometrical problem defining the shape. The former, extensively developed in this work, is most suitable for cases where an analytical solution of the geometrical problem is not possible. Multiple model variants were examined, using a custom MATLAB solver implemented to assess their performance against benchmark problems. While the models showed consistent performance in reproducing the anisotropic curvature driving force, discrepancies were found for the triple junction angle benchmark. Supported with these results, it was possible to select the most suitable model for the considered application. 

The analytical solution to the shape-defining problem, on the other hand, enabled the creation of shape factor-orientation maps. These were used in a Monte Carlo simulation algorithm to model polycrystal growth. This simulation demonstrated how anisotropic interface energy can influence texture evolution and lead to new texture evolution modes. These findings could offer a qualitative explanation for the peculiar texture evolution observed in electrodeposited nickel, providing new insights into the effects of anisotropic nucleation on material properties.




%%%%%%%%%%%%%%%%%%%%%%%%%%%%%%%%%%%%%%%%%%%%%%%%%%
% Keep the following \cleardoublepage at the end of this file, 
% otherwise \includeonly includes empty pages.
\cleardoublepage

% vim: tw=70 nocindent expandtab foldmethod=marker foldmarker={{{}{,}{}}}
