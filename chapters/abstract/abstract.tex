% !TeX root = ../../thesis.tex
\chapter{Abstract}                                 \label{ch:abstract}
This work investigates the effect of anisotropy in solid-liquid surface energy on orientation selection during repeated nucleation in polycrystalline film growth and the implications for the formation of crystallographic texture. For this purpose, a methodology was developed, which allowed to obtain the nucleation probability as function of the substrate and nucleus orientation. For every pair of these orientations, the nucleation probability was derived from the equilibrium stable shape of the nucleus at those conditions. Two approaches have been developed and validated to determine the shape: one based on the multi-phase field method and the other based on an analytical solution to the geometric problem. The first allowed to obtain the equilibrium shapes without any assumption on the shape itself, so it was applicable in the cases with non-existing analytic solution. A multi-phase field model incorporating anisotropic interface energy was extensively developed for this purpose.

Multiple model variants were investigated and an own solver for the governing equations was written in MATLAB. Two model-independent benchmark problems were developed to quantitatively compare performance of the developed models in terms of how they reproduced the anisotropic curvature driving force. Even though the models were comparably consistent in the above benchmarks, an additional benchmark problem concerning triple junction angles revealed thermodynamic inconsistency in two of the three models, causing them to be reliable only in isotropic or weakly anisotropic systems.

The analytic solution to the nucleus shape was used to obtain so-called shape factor-orientation maps, which were used in a nucleus orientation sampling algorithm. That one was implemented in Monte Carlo simulations of growing polycrystal and allowed to demonstrate how the nucleation with anisotropic interface energy may affect the course of texture evolution. New modes of texture evolution were identified and the insight was used to qualitatively explain an experimentally observed peculiar texture evolution in electrodeposited nickel. 


% RE-FORMULATED BY CHAT GPT
%This study investigates how anisotropy in solid-liquid surface energy influences orientation selection during repeated nucleation in polycrystalline film growth and its impact on crystallographic texture formation. A novel methodology was developed to determine the nucleation probability as a function of substrate and nucleus orientation, derived from the equilibrium shape of the nucleus under given conditions. Two approaches were devised to determine this shape: a multi-phase field method and an analytical geometric solution. The former, extensively developed for this work, is applicable in cases lacking an analytical solution. Multiple model variants were examined, with a custom MATLAB solver implemented to assess their performance against benchmark problems. While the models showed consistent performance in reproducing the anisotropic curvature driving force, discrepancies were found in triple junction angle benchmarks, limiting the reliability of two models to isotropic or weakly anisotropic systems.
%
%The analytic solution enabled the creation of shape factor-orientation maps, which were used in a Monte Carlo simulation algorithm to model polycrystal growth. This simulation demonstrated how anisotropic interface energy can influence texture evolution, leading to the identification of new texture evolution modes. These findings offer a qualitative explanation for the peculiar texture evolution observed in electrodeposited nickel, providing new insights into the effects of anisotropic nucleation on material properties.


%The first publication of the author quantitatively compared three multi-phase field models incorporating inclination-dependent interface energy and kinetic coefficient. Two model-independent benchmark problems were developed by the author for the purpose. One assessed the quality of match to Wulff shapes and the shrinkage rate of the shape and the second did the same but in a specific combination of the anisotropy in interface energy and kinetic coefficient, validating thus both features of the model at once. 
%Because the multi-phase field models are known to reproduce motion of the interfaces driven by curvature driving force, their typical application is grain growth (disregarding the possibility to include other physics in the model). From the perspective of model development, it is critical to assure correct representation of the force balance in the triple junctions. However, this is a non-trivial requirement due to numerical artifacts called ghost phases, which are in general hard to avoid in multi-phase field models. 
%Development of benchmark problems allowing quantification of multi-phase field model performance in triple junctions was thus recognized as highly relevant for the community. A collaboration was initiated with another scientific group at Karlsruhe Institute of Technology, Germany, which had expertise in a different type of multi-phase field models. In another two benchmarks focusing on triple junctions, a wide group of multi-phase field models was compared, and a publication was produced, co-authored by the author.
%The purpose of the herein developed models was to identify equilibrium stable shapes of heterogeneous nuclei with inclination-dependent interface energy as function of the nucleus and substrate crystallographic orientation. This knowledge can be used to derive the anisotropy in nucleation probability as function of the crystallographic orientations. Even though the multi-phase-field framework was developed enough to provide the results, the geometrical problem has an analytical solution.
%Knowledge of the analytic solution allowed to implement a much simpler and more efficient algorithm, providing deeper and more reliable insights. So-called shape factor-orientation maps were constructed with high resolution and were used as input in a Monte Carlo simulation of growing polycrystal to demonstrate how the nucleation with anisotropic interface energy may affect the course of texture evolution. New modes of texture evolution were identified and the insight was used to qualitatively explain an experimentally observed peculiar texture evolution in electrodeposited nickel. The latter findings were published in a third paper the author was involved in during his PhD research (as main author).



%%%%%%%%%%%%%%%%%%%%%%%%%%%%%%%%%%%%%%%%%%%%%%%%%%
% Keep the following \cleardoublepage at the end of this file, 
% otherwise \includeonly includes empty pages.
\cleardoublepage

% vim: tw=70 nocindent expandtab foldmethod=marker foldmarker={{{}{,}{}}}
