% !TeX root = ../../thesis.tex
\chapter{Extended multi-phase field model}
\label{ch_PFMoelans}
The multi-phase field model by Moelans~\cite{Moelans2008} is an established~\cite{Miyoshi2020} quantitative phase field model of grain growth with anisotropic grain boundary properties. Using asymptotic analysis, Moelans derived that in the original model~\cite{Moelans2008}, the local interface energy and width are related to three model parameters. Because these are two equations of three variables, the system is undetermined and one of the model parameters is free. This degree of freedom introduces the possibility of many different parameters assignment strategies, all of which represent the same physical input. The effect of different but equivalent parameters choices can thus be investigated in this model. Moelans proposed such parameters assignment strategy~\cite{Moelans2008}, which assured constant interface width irrespective of the strength of anisotropy in interface energy (a kind of \textit{natural} formulation). Two of the three model parameters ($\gamma$ and $\kappa$) were made anisotropic in order to achieve such behavior. However, this approach (here denoted IWc - \textit{Interface Width constant}) does not reproduce well the angles between interfaces in triple junctions for stronger anisotropies. This was already first noted by Moelans in~\cite{Moelans2010}. An alternative parameters assignment strategy with only parameter $\gamma$ anisotropic was used in~\cite{Ravash2017,Miyoshi2020}, but no systematic comparison was made. In this approach, the interface width is not constant in anisotropic systems, but it is not simply classifiable as \textit{classical} anisotropy formulation, because the gradient energy coefficient is constant. It will be denoted IWvG (\textit{Interface Width variable and Gamma anisotropic}). The third compared parameters assignment strategy is the \textit{classical} formulation, varying only gradient energy coefficient $\kappa$ to achieve the desired interface energy anisotropy (denoted IWvK - \textit{Interface Width variable and Kappa anisotropic}). Additionally, the inclination dependence of interface energy in IWvG and IWvK have not yet been addressed in the framework of Moelans' model. 


\section{Introduction}
The system consists of $n$ non-conserved continuous-field variables (further denoted phase fields) $\eta_1(\mathbf{r},t), \eta_2(\mathbf{r},t),\dots,\eta_n(\mathbf{r},t)$, which are functions of space and time. The total free energy of the system is expressed as functional of the phase fields and their gradients $\nabla\eta_1(\mathbf{r},t), \nabla\eta_2(\mathbf{r},t),\dots,\nabla\eta_n(\mathbf{r},t)$
\begin{equation} \label{eq_def_totalE_ver1}
	F = \int_V \Bigg\{ m f_0(\vec{\eta}) + \frac{\kappa}{2}\sum_{i=1}^n(\nabla \eta_i)^2 \Bigg\} \mathrm{d}V \,,
\end{equation}
where the homogeneous free energy density $f_0(\vec{\eta}) = f_0(\eta_1, \eta_2,\dots,\eta_n)$ is expressed as 
\begin{equation}
	f_0(\vec{\eta}) = \sum_{i=1}^n\left(\frac{\eta_i^4}{4} - \frac{\eta_i^2}{2} \right) +\gamma\sum_{i=1}^n\sum_{i>j}\eta_i^2\eta_j^2 + \frac{1}{4} \,.
\end{equation}
The parameters $m,\kappa, \gamma$ are model parameters, which together define interface energy and interface width (see the following section for more details).

The governing equations for each phase field $\eta_p$ are obtained based on the  functional derivative of the free energy functional with respect to $\eta_p$, assuming that the phase-fields are non-conserved, i.e. %(note that phase-field gradient components act as independent variables in the expression when taking the functional derivative), i.e. 
\begin{equation}
	\label{eq_ACeq_governing}
	\frac{\partial \eta_p}{\partial t} = -L\frac{\delta F}{\delta \eta_p} = -L \left[ \frac{\partial f}{\partial \eta_p} - \nabla\cdot\frac{\partial f}{\partial(\nabla \eta_p)} \right] \,,
\end{equation}
where $L$ is the kinetic coefficient (also dependent on the model parameters), $f$ is the full integrand in~\eqref{eq_def_totalE_ver1} and $\nabla\cdot\partial f/\partial(\nabla \eta_p)$ is divergence of vector field $\partial f/\partial(\nabla \eta_p)$ defined by relation
\begin{equation}
	\frac{\partial f}{\partial(\nabla \eta_p)} = \frac{\partial f}{\partial(\partial_x\eta_p)}\mathbf{n}_x + \frac{\partial f}{\partial(\partial_y \eta_p)}\mathbf{n}_y + \frac{\partial f}{\partial(\partial_z \eta_p)}\mathbf{n}_z
\end{equation}
with $\partial_x,\partial_y,\partial_z$ being operators for unidirectional derivatives in the corresponding directions and $\mathbf{n}_x,\mathbf{n}_y,\mathbf{n}_z$ coordinate base vectors. 

\section{Isotropic model}
\label{sec_Models}
In a system with uniform grain boundary properties, the interface energy is equal for all interfaces and hence the phase-field model parameters $m,\kappa, \gamma$ (and interface width $l$) are constant in the system.

Then, using the expression~\eqref{eq_ACeq_governing}, the governing equation for each phase field $\eta_p$ takes the following form % (assuming that none of $m,\kappa,\gamma$ is function of $\eta_p$ or of components of gradient $\nabla\eta_p$)
\begin{equation}
	\frac{\partial \eta_p}{\partial t} = -L\left[ m\left( \eta_p^3-\eta_p +  2\gamma\eta_p\sum_{j\neq p}\eta_j^2 \right) - \kappa\nabla^2\eta_p \right] 
\end{equation}

The interface energy $\sigma$ of the system is related to the model parameters via 
\begin{equation}\label{eq_IE}
	\sigma = g(\gamma)\sqrt{m\kappa} \,,
\end{equation}
where $g(\gamma)$ is a non-analytic function of parameter $\gamma$. The interface width $l$ is expressed as
\begin{equation} \label{eq_IW}
	l = \sqrt{\frac{\kappa}{mf_{0c}(\gamma)}} \,,
\end{equation}
where $f_{0c}(\gamma)$ is the value of $f_0(\eta_{i,cross},\eta_{j,cross})$ in the points where the two phase fields $\eta_i,\eta_j$ cross. $f_{0c}(\gamma)$ is a non-analytic function too. Values of both $g(\gamma)$ and $f_{0c}(\gamma)$ were tabulated and are available in~\cite{Ravash2017}. Both functions are positive and monotonously rising.

Usually, the interface energy $\sigma$ is known as material property and $l$ is chosen for computational convenience, together with $\gamma$. Then, the parameter values are assigned from the following formulae

\begin{equation}\label{eq_def_kappa}
	\kappa = \sigma l\frac{\sqrt{f_{0c}(\gamma)}}{g(\gamma)} \approx \frac{3}{4}\sigma l
\end{equation}
\begin{equation} \label{eq_def_m}
	m = \frac{\sigma}{l}\frac{1}{g(\gamma)\sqrt{f_{0c}(\gamma)}} \approx 6 \frac{\sigma}{l}
\end{equation}
\begin{equation}\label{eq_def_L}
	L = \frac{\mu}{l}\frac{g(\gamma)}{\sqrt{f_{0c}(\gamma)}} \approx \frac{4}{3} \frac{\mu}{l}
\end{equation}
The symbol $\mu$ stands for interface mobility. The approximate relations above hold exactly when $\gamma=1.5$ and are well applicable when $0.9 \leq \gamma \leq 2.65$ ~\cite{Moelans2008}. 

\section{Anisotropic model and parameters assignment strategies}
Two cases of interface energy anisotropy may occur, together or separately. Firstly, in the system there may be multiple interfaces with different interface energies (termed \textit{misorientation dependence} in~\cite{Moelans2008}, here \textit{pair-wise isotropy} for greater generality). Secondly, there may be an interface with inclination-dependent interface energy. Additionally, the kinetic coefficient $L$ can be inclination dependent. 

In both cases of anisotropy in interface energy, some of the model parameters $m,\kappa, \gamma$ must become spatially dependent in order to assure correct local representation of the interface energy and width. In other words, the equations~\eqref{eq_IE}~and~\eqref{eq_IW} must hold in every point of the anisotropic system.
These two equations locally form an undetermined system of three variables, hence one of the model parameters is free and many different parameters assignment strategies are possible.

In this paper, the parameter $m$ is always a constant, because when $m$ was spatially varied~\cite{Moelans2008}, the model behavior in multijunctions was reported to be strongly affected by the interface width. Such model would be non-quantitative and thus will not be further regarded in this paper.

Three different parameters assignment strategies are considered, which differ in value of $m$ and further in which of parameters $\kappa,\gamma$ is constant and which varies in space to keep equation~\eqref{eq_IE} valid. The three strategies are denoted: IWc (variable $\gamma,\kappa$ so that interface width is constant  \cite{Moelans2008}), IWvG (variable interface width and $\gamma$ \cite{Ravash2017}) and IWvK (variable interface width and $\kappa$).   Table~\ref{tab_models_comparison} summarizes, which parameters are kept constant and which vary to capture the anisotropy in the different strategies. The detailed procedure of the parameters assignment and ways to control the width of the narrowest interface are described in Appendix~\ref{sec_suppl_param_det}. Note that for the IWc model we propose a single-step parameters determination procedure, which is more predictable and simpler than the original iterative one~\cite{Moelans2008}. The two are equivalent, though.\\
Below follow details about the incorporation of pair-wise isotropy and inclination-dependence in the model. 

\begin{table}[h]
	\centering
	\caption{Characterization of the three parameter assignment strategies: the one with constant interface width (IWc), with variable interface width and all anisotropy in $\gamma$ (IWvG) and with variable interface width and all anisotropy in $\kappa$ (IWvK). In the latter, it is inconvenient to choose other value of $\gamma$ than $\gamma=1.5$. IW stands for interface width, other symbols have meaning as in the text.}
	\label{tab_models_comparison}
	\begin{tabular}{l|c|c|c|}\footnotesize
		& IWc & IWvG & IWvK \\ \hline
		fixed parameters & IW, $m$ & $\kappa, m$ & $\gamma, m$  \\
		varying parameters & $\gamma, \kappa$ & IW, $\gamma$ & IW, $\kappa$
	\end{tabular}
\end{table}

\subsection{Systems with pair-wise isotropic IE}
In the system with $n$ phase fields, there are $n(n-1)/2$ possible pair-wise interfaces, each of which may have different (mean) interface energy $\sigma_{i,j}$. The indices $i,j$ denote interface between phase fields $\eta_i,\eta_j$. A set of parameters $m,\kappa_{i,j},\gamma_{i,j}, L_{i,j}$ (all scalars) is obtained by appropriate procedure  (depending on the strategy, see Appendix~\ref{sec_suppl_param_det}) so that the relations \eqref{eq_IE} and \eqref{eq_IW} are valid for each interface independently (equations~\eqref{eq_def_kappa}-\eqref{eq_def_L} hold for each interface ($i$-$j$)). Then, these are combined together to produce the model parameter fields $\kappa(\bm{r}),\gamma(\bm{r}),L(\bm{r})$:
\begin{equation}
	\kappa(\bm{r}) = \frac{\sum_{i=1}^n\sum_{j>i}^n\kappa_{i,j}\eta_i^2\eta_j^2}{\sum_{i=1}^n\sum_{j>i}^n\eta_i^2\eta_j^2}
\end{equation}
\begin{equation}
	\gamma(\bm{r}) = \frac{\sum_{i=1}^n\sum_{j>i}^n\gamma_{i,j}\eta_i^2\eta_j^2}{\sum_{i=1}^n\sum_{j>i}^n\eta_i^2\eta_j^2} \,,
\end{equation}
\begin{equation}
	L(\bm{r})= \frac{\sum_{i=1}^n\sum_{j>i}^nL_{i,j}\eta_i^2\eta_j^2}{\sum_{i=1}^n\sum_{j>i}^n\eta_i^2\eta_j^2}
\end{equation}

which stand in place of $\kappa,\gamma$ in the functional in equation~\eqref{eq_def_totalE_ver1} and in place of $L$ in the governing equation~\eqref{eq_ACeq_governing}.

Notice that from Table~\ref{tab_models_comparison} stems  that in IWvG is $\kappa(\bm{r})=\mathrm{const}$ by definition (i.e. all the $\kappa_{i,j}$s are equal) and similarly in IWvK $\gamma(\bm{r})=\mathrm{const}=1.5$ (i.e. all $\gamma_{i,j}$s are equal). 

The free energy functional for IWc is then (those for IWvG and IWvK are equal, only with either $\kappa(\bm{r})$ or $\gamma(\bm{r})$ being constants, respectively):
\begin{equation} \label{eq_def_totalE_ver2}
	F = \int_V \Bigg\{ m f_0(\vec{\eta}) + \frac{\kappa(\bm{r})}{2}\sum_{i=1}^n(\nabla \eta_i)^2 \Bigg\} \mathrm{d}V \,,
\end{equation}
\begin{equation}
	f_0(\vec{\eta}) = \sum_{i=1}^n\left(\frac{\eta_i^4}{4} - \frac{\eta_i^2}{2} \right) +\sum_{i=1}^n\sum_{i>j}\gamma_{i,j}\eta_i^2\eta_j^2 + \frac{1}{4} \,.
\end{equation}

Both parameter fields $\kappa(\bm{r}),\gamma(\bm{r})$ are functions of phase fields $\vec{\eta}$. This dependence should produce new terms in the governing equations (from $\partial f/\partial \eta_p$ in equation~\eqref{eq_ACeq_governing}). However, because the denominator of $\gamma(\bm{r})$ cancels out in the functional, the new terms only arise from $\partial \kappa/\partial \eta_p$.\\
The governing equations then are 
\begin{equation}
	\begin{split}
		\frac{\partial \eta_p}{\partial t} = -L(\bm{r})\left[ m\left( \eta_p^3-\eta_p +  2\eta_p\sum_{j\neq p}\gamma_{p,j}\eta_j^2 \right) \right. \\ \left. +\frac{1}{2}\frac{\partial \kappa}{\partial\eta_p}\sum_{i=1}^n(\nabla\eta_i)^2 - \kappa(\bm{r})\nabla^2\eta_p \right] 
	\end{split}
\end{equation}

The above procedure is fully variational, nevertheless inclusion of the term proportional to $\partial \kappa/\partial \eta_p$ enables the model to reduce the total energy of the system by introduction of so called third phase contributions (also ghost or spurious phases) at diffuse interfaces.~\cite{Moelans2008} That is a common problem in multi-phase field models~\cite{Toth2015}, where a third phase field attains non-zero value within an interface of two other phase fields. This mathematical artefact affects triple junction angles and in general is not physically justified. Several ways of elimination or suppression of ghost phases were described in~\cite{Toth2015} and the references therein.

In this work, the ghost phases were eliminated by neglecting the term proportional to $\partial \kappa/\partial \eta_p$. However, because such model is not fully variational, the thermodynamic consistency can no longer be guaranteed in IWc and IWvK. This does not affect IWvG, because there is $\partial \kappa/\partial\eta_p=0$ anyway. That accounts for a clear advantage of the IWvG model, as no ghost phases appear even when fully variational.

\subsection{Systems with inclination-dependent interface energy} \label{sec_model_incldepIE}
\begin{sidewaystable}[]
	\centering
	\caption{Inclination dependence of the variable parameters in the respective models. The interface energy is $\sigma_{i,j}(\theta_{i,j})=\sigma_{i,j}^0h_{i,j}(\theta_{i,j})$. Symbols $\kappa_{i,j}^0, \gamma_{i,j}^0$ stand for scalar values of the parameters determined from $\sigma_{i,j}^0$ (see~\ref{sec_suppl_param_det} in~\cite{Minar2021suppl}). Expressions for $\gamma_{i,j}(\theta_{i,j})$ follow the so called \textit{weak anisotropy approximation}~\cite{Moelans2008}, i.e. they assume that the values of $\gamma_{i,j}(\theta_{i,j})$ do not diverge far from 1.5, so that the approximation $g^2[\gamma_{i,j}(\theta_{i,j})]\approx16[2\gamma_{i,j}(\theta_{i,j})-1]/9[2\gamma_{i,j}(\theta_{i,j}) +1]$ is applicable (see~\cite{Moelans2008} for details). Second row contains expressions used in equations~\ref{eq_dkppijdGp} and \ref{eq_dgmmijdGp}.}
	\label{tab_models_comp__par_incldep}
	\begin{tabular}{p{2cm}|>{\centering\arraybackslash}p{5cm}>{\centering\arraybackslash}p{5cm}>{\centering\arraybackslash}p{4.2cm}}
		\toprule
		model     & IWc & IWvG & IWvK  \\ \hline
		variable parameter(s)     & $\begin{array}{l}
			\kappa_{i,j}(\theta_{i,j})= \kappa_{i,j}^0h_{i,j}(\theta_{i,j}) \\
			\gamma_{i,j}(\theta_{i,j})= -\frac{\frac{9}{4}g^2(\gamma_{i,j}^0)h_{i,j}(\theta_{i,j})+1}{\frac{9}{2}g^2(\gamma_{i,j}^0)h_{i,j}(\theta_{i,j})-2}
		\end{array}$
		& $\gamma_{i,j}(\theta_{i,j})= -\frac{\frac{9}{4}[g(\gamma_{i,j}^0)h_{i,j}(\theta_{i,j})]^2+1}{\frac{9}{2}[g(\gamma_{i,j}^0)h_{i,j}(\theta_{i,j})]^2-2}$
		&
		$\kappa_{i,j}(\theta_{i,j})= \kappa_{i,j}^0[h_{i,j}(\theta_{i,j})]^2$ \\ \hline
		$\partial \kappa_{i,j}/\partial h_{i,j}$ and $\partial \gamma_{i,j}/\partial h_{i
			,j}$ 
		& $\begin{array}{l}
			\partial \kappa_{i,j}/\partial h_{i,j} = \kappa_{i,j}^0 \\
			\partial \gamma_{i,j}/\partial h_{i,j} = \frac{9g^2( \gamma_{i,j}^0)}{\left[\frac{9}{2}g^2(\gamma_{i,j}^0)h_{i,j}(\theta_{i,j}) - 2\right]^2}
		\end{array}$ 
		& $\partial \gamma_{i,j}/\partial h_{i,j} = \frac{18g^2(\gamma_{i,j}^0)h_{i,j}(\theta_{i,j})}{\left\{\frac{9}{2}[g(\gamma_{i,j}^0)h_{i,j}(\theta_{i,j})]^2 - 2\right\}^2}$ 
		& $\partial \kappa_{i,j}/\partial h_{i,j} = 2 \kappa_{i,j}^0h_{i,j}(\theta_{i,j})$\\
		\bottomrule
	\end{tabular}
\end{sidewaystable}
The orientation of an interface in 2D system is given by interface normal, inclined under the angle $\theta$. Local value of interface energy may be a function of local interface inclination, i.e. $\sigma = \sigma(\theta)$. In Moelans' model~\cite{Moelans2008}, the normal at interface between $\eta_i,\eta_j$, denoted $\hat{\bm{n}}_{i,j}$, is defined as 
\begin{equation} \label{eq_def_inclination}
	\hat{\bm{n}}_{i,j} = \frac{\nabla\eta_i-\nabla\eta_j}{|\nabla\eta_i-\nabla\eta_j|} = \left[\begin{array}{c}
		(\hat{n}_{i,j})_x   \\
		(\hat{n}_{i,j})_y
	\end{array} \right]
\end{equation}
and the definite inclination of that normal 
\begin{equation}
	\theta_{i,j} = \mathrm{atan2}[(\hat{n}_{i,j})_y,(\hat{n}_{i,j})_x] \,,
\end{equation}
which is the standard 2-argument arctangent function.\\
In 2D, the inclination-dependence of interface energy can be expressed as 
\begin{equation}\label{eq_IE_incldep}
	\sigma_{i,j}(\theta_{i,j}) = \sigma_{i,j}^0h_{i,j}(\theta_{i,j})
\end{equation}
where $\sigma_{i,j}^0$ is a scalar and  $h_{i,j}(\theta_{i,j})$ is anisotropy function. The used anisotropy function was
\begin{equation}
	h_{i,j}(\theta_{i,j}) = 1 + \delta\cos(n\theta_{i,j}) \,,
\end{equation}

with $\delta$ being strength of anisotropy and $n$ the order of symmetry. Some properties of this anisotropy function and the resulting Wulff shapes are given in~\ref{sec_appendix_anisofun_props}. 

The inclination dependence of $\sigma_{i,j}$ implies that some of the model parameters $\gamma_{i,j},\kappa_{i,j}$ must be taken inclination-dependent too. Depending on the model used (IWc, IWvG or IWvK), the local validity of equation~\eqref{eq_IE} is achieved using different inclination dependence of the variable parameters (see Table~\ref{tab_models_comp__par_incldep} for details).

Because the inclination-dependent $\kappa_{i,j},\gamma_{i,j}$ are functions of components of gradients $\nabla\eta_i,\nabla\eta_j$, the divergence term in the functional derivative (equation~\eqref{eq_ACeq_governing}) produces additional driving force terms. In the general case with multiple inclination-dependent interfaces, the divergence term equals
\begin{equation}\label{eq_incldep_divDFterms}
	\begin{split}
		\nabla\cdot \frac{\partial f}{\partial(\nabla \eta_p)} &= 2m\eta_p\nabla\eta_p\cdot \left[\sum_{j\neq p}  \eta_j^2\frac{\partial \gamma_{p,j}}{\partial (\nabla\eta_p)}\right]  \\ 
		&\quad+ 2m\eta_p^2\sum_{j\neq p} \left[\eta_j\nabla\eta_j \cdot \frac{\partial \gamma_{p,j}}{\partial (\nabla\eta_p)}\right] \\
		&\quad+ m\eta_p^2\sum_{j\neq p}\eta_j^2\left[\nabla\cdot\frac{\partial \gamma_{p,j}}{\partial (\nabla\eta_p)}\right] \\
		&\quad + \frac{1}{2}\left[\nabla\cdot\frac{\partial \kappa}{\partial (\nabla\eta_p)}\right]\sum_{i=1}^n(\nabla \eta_i)^2 \\
		&\quad+ \frac{1}{2}\frac{\partial \kappa}{\partial (\nabla\eta_p)}\cdot\left[\nabla\sum_{i=1}^n(\nabla \eta_i)^2\right]  \\
		&\quad+ \nabla\kappa(\bm{r})\cdot\nabla\eta_p + \kappa(\bm{r})\nabla^2\eta_p \,.
	\end{split} 
\end{equation}
The vector field $\partial \kappa/\partial(\nabla\eta_p)$ is
\begin{equation} \label{eq_dkppdGp}
	\frac{\partial \kappa}{\partial(\nabla\eta_p)} =  \frac{\sum\limits_{j\neq p}^n \left(\frac{\partial \kappa_{p,j}}{\partial (\nabla\eta_p)}\right)\eta_p^2\eta_j^2}{\sum\limits_{k=1}^n\sum\limits_{l>k}\eta_k^2\eta_l^2} \,
\end{equation}
where the sum in the numerator goes through all pair-wise interfaces of $\eta_p(\bm{r})$. The vector fields $\partial\kappa_{p,j}/\partial (\nabla\eta_p)$ are expressed
\begin{equation} \label{eq_dkppijdGp}
	\frac{\partial \kappa_{p,j}}{\partial (\nabla\eta_p)} = \frac{1}{|\nabla\eta_i-\nabla\eta_j|}\frac{\partial \kappa_{p,j}}{\partial h_{p,j}}\frac{\partial h_{p,j}}{\partial \theta_{p,j}} \left[\begin{array}{c}
		-(\hat{n}_{i,j})_y   \\
		(\hat{n}_{i,j})_x
	\end{array} \right].
\end{equation}
Note, that the above vector field is nonzero only in IWc and IWvK models at the interfaces ($p$-$j$) with inclination-dependent IE. Likewise, the below vector field is nonzero only in IWc and IWvG
\begin{equation} \label{eq_dgmmijdGp}
	\frac{\partial \gamma_{p,j}}{\partial (\nabla\eta_p)} = \frac{1}{|\nabla\eta_i-\nabla\eta_j|}\frac{\partial \gamma_{p,j}}{\partial h_{p,j}}\frac{\partial h_{p,j}}{\partial \theta_{p,j}} \left[\begin{array}{c}
		-(\hat{n}_{i,j})_y   \\
		(\hat{n}_{i,j})_x
	\end{array} \right].
\end{equation}
The multipliers $\partial \kappa_{p,j}/\partial h_{p,j}$ and $\partial \gamma_{p,j}/\partial h_{p,j}$ differ in individual models and are also provided in Table~\ref{tab_models_comp__par_incldep}. The term $\partial h_{p,j}/\partial \theta_{p,j}$ is defined by the inclination-dependence at the interface ($p$-$j$).

The governing equation then is
\begin{equation}
	\begin{split}
		\frac{\partial \eta_p}{\partial t} = -L(\bm{r})\left[ m\left( \eta_p^3-\eta_p +  2\eta_p\sum_{j\neq p}\gamma_{p,j}(\theta_{p,j})\eta_j^2 \right) \right. \\ 
		\left.  -\nabla\cdot \frac{\partial f}{\partial(\nabla \eta_p)} \right] \,.
	\end{split}
\end{equation}

Note, that the term proportional to $\partial \kappa/\partial\eta_p$ was neglected here. 

In models with variable interface width (IWvG, IWvK), at the interfaces with inclination-dependent interface energy, the interface width is a function of the inclination, i.e. $l_{i,j}=l_{i,j}(\theta_{i,j})$. Because the kinetic coefficient $L_{i,j}$ is inversely proportional to the interface width $l_{i,j}$ (see equation~\eqref{eq_def_L}), the kinetic coefficient is inclination-dependent as well (even for \textit{constant} grain boundary mobility $\mu_{i,j}$). The inclination dependence of $L_{i,j}(\theta_{i,j})$ due to interface width variation is in the IWvG model
\begin{equation}\label{eq_Lcorr_IWvG}
	L_{i,j}(\theta_{i,j}) = L_{i,j}h_{i,j}(\theta_{i,j})
\end{equation}
and in the IWvK model
\begin{equation}\label{eq_Lcorr_IWvK}
	L_{i,j}(\theta_{i,j}) = L_{i,j}/h_{i,j}(\theta_{i,j}) \,,
\end{equation}
where $h_{i,j}(\theta_{i,j})$ is the anisotropy function in interface energy~\eqref{eq_IE_incldep}.% $\sigma_{i,j}(\theta_{i,j})=\sigma_{i,j}^0h_{i,j}(\theta_{i,j})$.

The equations~\eqref{eq_Lcorr_IWvG},\eqref{eq_Lcorr_IWvK} were derived from an alternative expression for the kinetic coefficient $L_{i,j}$
\begin{equation}
	L_{i,j} = \frac{\mu_{i,j}\sigma_{i,j}(\theta_{i,j})}{\kappa_{i,j}(\theta_{i,j})} \,,
\end{equation}
where the inclination dependencies of the right-hand side were expressed correspondingly to the model (see Table~\ref{tab_models_comp__par_incldep} for $\kappa_{i,j}(\theta_{i,j})$).

Due to varying number of driving force terms in the three parameter assignment strategies, the governing equations are different in each and hence it is justified to call them different models.

\subsubsection{Systems with inclination-dependent mobility}
Let the interface ($i$-$j$) have isotropic interface energy and inclination-dependent grain boundary mobility with anisotropy function $h_{i,j}^\mu(\theta_{i,j})$, i.e. $\mu_{i,j}=\mu_{i,j}(\theta_{i,j})=\mu_{i,j}^0h_{i,j}^\mu(\theta_{i,j})$. From equation~\eqref{eq_def_L} we can see that the kinetic coefficient must have the same anisotropy, i.e. $L_{i,j}(\theta_{i,j})=L_{i,j}^0h_{i,j}^\mu(\theta_{i,j})$, where $L_{i,j}^0=\mu_{i,j}^0g(\gamma_{i,j})/l_{i,j}f_{0c}(\gamma_{i,j})$. 

If the interface energy is inclination-dependent as well and a model with variable interface width is used (either IWvG or IWvK), the inclination dependence in $L_{i,j}(\theta_{i,j})$ due to the interface width variation must be included similarly like in \eqref{eq_Lcorr_IWvG} and \eqref{eq_Lcorr_IWvK}. The physical inclination-dependence is independent from the one due to interface width variation, implying the following expression for IWvG model
\begin{equation}     
	L_{i,j}(\theta_{i,j}) = L_{i,j}^0h_{i,j}(\theta_{i,j})h_{i,j}^\mu(\theta_{i,j})
\end{equation}
and for the IWvK model analogically
\begin{equation}
	L_{i,j}(\theta_{i,j}) = L_{i,j}^0\frac{h_{i,j}^\mu(\theta_{i,j})}{h_{i,j}(\theta_{i,j})} \,,
\end{equation}
where $h_{i,j}(\theta_{i,j})$ is the interface energy anisotropy function.

\subsection{Interface profiles in different models} \label{sec_difference_in_profiles}
The main difference in the model modifications is how the interface width varies as function of local interface energy. Obviously, in IWc the width is constant. In IWvK (with $\gamma=1.5$) the width of interface $i$-$j$ can be computed as
\begin{equation}
	l_{i,j} = 6\frac{\sigma_{i,j}}{m} \,.
\end{equation}
Apparently, in IWvK model the interface width is proportional to the interface energy, i.e. the larger the interface energy, the larger the interface width.

In IWvG the width can be expressed from~\eqref{eq_IW} and~\eqref{eq_IE} assuming $\frac{4}{3}\sqrt{f_{0,c}}(\gamma_{i,j})=g(\gamma_{i,j})$ (which holds for small values of $\gamma_{i,j}$). Then, it goes approximately
\begin{equation}
	l_{i,j} \approx \frac{\kappa}{\sigma_{i,j}} \,,
\end{equation}
and apparently the larger interface energies are associated with lower interface widths in IWvG. 


\section{General Neumann boundary conditions to control interface inclination at domain boundary}
This model extension was inspired by~\cite{Granasy2007}. It allows to control the interface inclination angle $\phi$ at the boundary, which thus becomes an input parameter. The principle is explained using a single phase field $\eta(\bm{r})$. 
First, we remark that the Neumann BCs $\nabla\eta\cdot\bm{n_D}=0$ must be a special case of the general BC ($\bm{n_D}$ is the domain boundary normal). The Neumann BC implies perpendicularity of the PF gradient and domain boundary normal. In other words, the interface is perpendicular to the domain boundary. Generally, we can write $\bm{n_D}\cdot \nabla\eta=|\nabla\eta|\cos(\phi)$. In the model by Granasy~\cite{Granasy2007}, the local magnitude of phase field gradient may be expressed using the local phase field value, giving rise to expression of the boundary condition as
\begin{equation}
	\bm{n_D}\cdot \nabla\eta=\frac{\cos(\phi)}{\delta\sqrt{2}}\eta(1-\eta)\,,
\end{equation}
where $\delta$ is the interface width. At the domain boundary, the gradient $\nabla\eta$ is expressed as a polynomial, which allows straightforward implementation of the condition, especially in rectangular simulation domains.\\

In Moelan's model the interface normal between the phase fields $\eta_i,\,\eta_j$ denoted $\bm{n}_{i,j}$ is defined as local difference in neighboring PF gradients (see equation~\ref{eq_def_inclination}). In the spirit of the introductory example we can write
\begin{equation}
	\bm{n_D}\cdot\bm{n}_{i,j} = \cos(\phi) \quad \implies \quad \bm{n}\cdot(\nabla\eta_i-\nabla\eta_j) = |\nabla\eta_i-\nabla\eta_j|\cos(\phi) \,.
\end{equation}
Now we have single BC with the correct physical interpretation (fixed inclination angle at the boundary) but because the governing equations are solved for the phase fields, a set of equivalent boundary conditions (one for each phase field) must be derivable from the above. In these independent BC, it must be possible to express the gradient magnitude (the introductory example finds a polynomial of local PF values equal to the gradient magnitude). \\
The first problem (coupling in the BC) would be solved if one gradient could be written as a function of the other and vice versa. However, this dependence is non-analytical for all $\gamma_{i,j}\neq1.5$. For $\gamma_{i,j}=1.5$ it is simply $\eta_i = 1-\eta_j$ and therefore also $\nabla\eta_i=-\nabla\eta_j$.~\cite{Moelans2008} \\
The second problem (the expression for gradient magnitude) is also enabled by the choice $\gamma=1.5$, as in such case there were derived analytic expressions for the gradients $\nabla\eta_i,\nabla\eta_j$.~\cite{Moelans2008} In 1D system with 2 phase fields holds
\begin{equation} \label{eq_PFgradient_analytic}
	\begin{split}
		\frac{\mathrm{d}\eta_i}{\mathrm{d}x} &= \sqrt{\frac{2m}{\kappa_{i,j}}}\eta_i(1-\eta_i) \overset{\mathrm{2D, 3D}}{=} |\nabla\eta_i| \\
		\frac{\mathrm{d}\eta_j}{\mathrm{d}x} &= \sqrt{\frac{2m}{\kappa_{i,j}}}\eta_j(1-\eta_j) \overset{\mathrm{2D, 3D}}{=} |\nabla\eta_j| \,.
	\end{split}
\end{equation}
Validity of the last equation sign in 2D and 3D (in systems with 2 phase fields) can be checked using the expressions in~\cite{Moelans2008} as follows. Write the equation 7 from~\cite{Moelans2008} for 2D. From it one gets 2D equivalents of equations 8a and 8b, where $\frac{\mathrm{d}\eta}{\mathrm{d}x}$ is replaced by $|\nabla\eta|$. Further, because the equation 19 does not depend on problem dimensionality, we obtain here the expressions in equation~\ref{eq_PFgradient_analytic}. In systems with more than two phase fields the general expression for the gradient magnitude is more complex. However, the same relations are satisfied at a pair-wise interface not close to the triple junction. \\
With the above and assuming that there is no triple junction near the domain boundary, we can thus write
\begin{equation}
	\begin{split}
		\bm{n}\cdot(\nabla\eta_i-\nabla\eta_j) &= 2\bm{n}\cdot\nabla\eta_i = 2|\nabla\eta_i|\cos(\phi) \\ &= -2\bm{n}\cdot\nabla\eta_j = -2|\nabla\eta_j|\cos(\phi) \,,
	\end{split}
\end{equation}
which provides very similar BC like in the introductory example for each of the two phase fields.\\
The only way how to use these relations is to have $\gamma_{i,j}=1.5$. In the case of inclination-dependent interface energy $\sigma_{i,j}(\theta_{i,j})=\sigma_{i,j}^0h_{i,j}^\sigma(\theta_{i,j})$, it is necessary to express the anisotropy fully by the parameter $\kappa_{i,j}(\theta_{i,j})$. 

\section{Volume conserving multi phase field models}
There are several approaches which accomplish the volume conservation of some species or phases. Each has its drawbacks though. It was not clear which of the possible approaches was the most convenient for coupling with inclination-dependent interface energy in curvature-driven systems.

Three conceptual solutions can be used in a multi-phase field model, which are known to the author: Cahn-Hilliard equation, Lagrange multipliers and fictitious concentrations field. 

Given the intended application with anisotropic interface energy and the complicated driving force, a less computationally demanding option than the Cahn-Hilliard equation was seeked, because that one is a partial differential equation of 4-th order in space. 

The approach using Lagrange multipliers to conserve volume was thoroughly considered, but eventually it was found out that in Moelan's model is not a suitable solution. The reason is that the volume of a single phase/grain is defined using all other phase fields, which introduces excessive computational complexity in application of the principles leading to volume conservation. These findings were documented in Appendix~\ref{ch_lagrange_multipliers_PF}.

The approach using fictitious concentration field is not new in combination with Moelans' multi-phase field model (see e.g.~\cite{Yadav2016,Yadav2018vol_cons}), but it will be reviewed.

\subsection{Fictitious concentration field}
The idea is to couple the AC equation to conserved concentration field in such a way that the change in phase fraction is only possible with exchange of species between phases. If the concentration of the independent species is set to equilibrium value no phase transformation occurs and the volume of the phase fraction should be conserved. 

In the description below, it is assumed that there are two phases, one solid and the other liquid.

The free energy functional is then
\begin{equation}
	F_{cons} = \int_V \left[ f_{hom}(\vec{\eta},\nabla\vec{\eta},c) + f_{grad}(\vec{\eta},\nabla\vec{\eta}) \right] \mathrm{d}V \,,
\end{equation}
with   
\begin{align}
	f_{hom}(\vec{\eta},\nabla\vec{\eta},c)  &= f_0(\vec{\eta},\nabla\vec{\eta}) + f_{chem}(\vec{\eta},c) \\
	f_{chem}(\vec{\eta},c) &= h_S(\vec{\eta})f_S(c) + h_L(\vec{\eta})f_L(c)
\end{align}
$h_S(\vec{\eta}), h_L(\vec{\eta})$ are solid and liquid interpolation functions. Assuming that the solid phase is composed of $s$ phase fields $\eta_{S1},\eta_{S2},\dots,\eta_{Ss}$ and the liquid phase of $l$ phase fields $\eta_{L1},\eta_{L2},\dots,\eta_{Ll}$ we can write
\begin{align}
	h_S(\vec{\eta}) &= \frac{\sum_i^s \eta_{Si}^2}{\sum_i^s \eta_{Si}^2 + \sum_j^l \eta_{Lj}^2}   \\
	h_L(\vec{\eta}) &= \frac{\sum_j^l \eta_{Lj}^2}{\sum_i^s \eta_{Si}^2 + \sum_j^l \eta_{Lj}^2} \quad .
\end{align}
Parabolic energy approximation will be used for the concentration dependence of the homogeneous phase free energy densities $f_S(c), f_L(c)$ as
\begin{align}
	f_S(c_S) &= A(c_S-c_{S,eq})^2 \\
	f_L(c_L) &= B(c_L-c_{L,eq})^2 \,.
\end{align}
An ideal solution approximation is adopted.

The governing equations for two phase fields are then
\begin{align}
	\frac{\partial \eta_1}{\partial t} &= -L\left[\frac{\partial f_0}{\partial \eta_1} - \nabla\cdot \frac{\partial F}{\partial (\nabla\eta_1)} + \frac{\partial h_S}{\partial \eta_1}[f_S(c_S) - f_L(c_L) - (c_S-c_L)\mu] \right] \\
	\frac{\partial \eta_2}{\partial t} &= -L\left[\frac{\partial f_0}{\partial \eta_2} - \nabla\cdot \frac{\partial F}{\partial (\nabla\eta_2)} + \frac{\partial h_S}{\partial \eta_2}[f_S(c_S) - f_L(c_L) - (c_S-c_L)\mu] \right] \\
	\mu &= 2A(c_L-c_{L,eq})\\
	\frac{\partial c}{\partial t} &= \frac{D}{A}\Delta\mu  = 2D\Delta(c_L-c_{L,eq})
\end{align}
where the diffusion coefficient $D$ is a constant.

The interpolation functions have the following property
\begin{equation}
	\frac{\partial h_S}{\partial \eta_p} = -\frac{\partial h_L}{\partial \eta_p}
\end{equation}

%The difference in grand potential in the solid and liquid phase is the driving force for change of the phase fraction. It is assumed that $A=B$ and thus
%\begin{equation}
%	f_S(c_S) - f_L(c_L) - (c_S-c_L)\mu = 
%\end{equation}





\section{Summary and the author's contribution}
The author extended the multi-phase field model by Moelans~\cite{Moelans2008}. The work~\cite{Moelans2008} promoted usage of the model variant with constant interface width (here denoted IWc) with not-fully-variational formulation and described it in great detail.

At the beginning of the author's work, it was already known that IWc was not reliably reproducing the angles in triple junctions of interfaces with different interface energies~\cite{Moelans2010_thinfilm} (i.e. in pair-wise isotropic systems). The model variant with variable interface width and all anisotropy in the parameter $\gamma$ (here IWvG) was used in one grain growth study~\cite{Ravash2017}, but its better reliability in triple junction angles had not been documented yet.

As provided below with all details, the model extensions included
\begin{enumerate}
	\item Description of another model variant with all anisotropy in the parameter $\kappa$ and variable interface width (IWvK)
	\item Derivation of the governing equation for all the three model variants in both 2D and 3D for the case of inclination-dependent interface energy
	\item The model behavior depends heavily on proper parametrization of the equations. Best practices in parameters determination were  developed and described (different in each model variant) in order to assure control over both the local interface width and local interface energy. 
	\item In the IWvK model variant such boundary conditions were incorporated, which allowed to control the angle under which the interface intersecting the domain boundary will align.
\end{enumerate}

Besides the listed model extensions, the author also implemented two approaches adding the volume conservation to the system based on Allen-Cahn type of equations, namely one using Lagrange multipliers to conserve the volume and pne using fictitious concentration field. Volume conservation would be necessary in simulations, where the nucleus interacted dynamically with its surroundings, but these simulations were not needed eventually. Regardless, the model modifications including volume conservation were described in Appendix.

%%%%%%%%%%%%%%%%%%%%%%%%%%%%%%%%%%%%%%%%%%%%%%%%%%
% Keep the following \cleardoublepage at the end of this file, 
% otherwise \includeonly includes empty pages.
\cleardoublepage

% vim: tw=70 nocindent expandtab foldmethod=marker foldmarker={{{}{,}{}}}
