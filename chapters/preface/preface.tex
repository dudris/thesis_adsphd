% !TeX root = ../../thesis.tex
\chapter*{Preface}                                  \label{ch:preface}
\section*{Preface}
My PhD journey started in May 2018. I was eager to take the challenge of starting in an unfamiliar, yet impactful field of research at one of the most prestigious universities in Europe. My enthusiasm was fueled by my innate curiosity and newly found courage into science I got after doing my thesis at Institute of Plasma Physics, Czech Academy of Sciences, Prague. Having just defended the prize-awarded thesis made me feel like I could do more than I used to believe. Most of the coming six and a half years I would live in deep fear that this time the hurdle has gone way too high...

The freshly engaged boy with bright eyes, who arrived to Leuven back then had to grow into a much stronger person while taking the many life lessons this PhD experience came with. The pressure of the PhD study in Leuven is enormous and way harder than anything I experienced before, for reasons which do not necessarily relate to high academic standards. The stereotypical solitude of the PhD student standing in front of this grandiose ambition of making a scientific breakthrough from a professionally junior position was strongly underlined by the isolation we all went through in corona. That was when my second year was ending. This marked also the moment, when I no longer was in Leuven most of the time, as me and my wife begun spending more time in Czechia. After another year and a half our daughter was born and we already knew that Czechia would be our home.

Of the six and a half years, I thus had about the first two years to live a somewhat standard PhD life. Not so distant past, yet a completely different life. Many aspects of it I enjoyed dearly. The intellectual freedom, the privilege to dig deeper into the state-of-the art science, all the discussions and learning. The first big conference without any stress. Small moments in the beautiful botanical garden or the inspiring ambience of the cool libraries and study rooms, where one can feel the serious presence of the many books filled with knowledge. New perspectives, friendships, food, and search for what feels good in my life. I am proud to be a KU Leuven alumnus and certainly much richer thanks to he experience.

I already knew I could work hard and it used to bring results in the past. Yet still, for some reason, my PhD always felt like I was lagging one step behind, if not more. Especially after the first year when I decided to switch the topic, my initial perceived self-confidence was gone completely. I lived through several nervous breakdowns and sincerely hated every May because that was when the PhD milestones were to be passed and I simply was one year behind the schedule. It was especially in these moments when I could fully rely on my supervisor, Prof. Nele Moelans. Proofreading of both my papers and this thesis, as well as milestone reports or grant application was invaluable to my PhD. I am grateful for your trust, which led you to choose me as one of students under your INTERDIFFUSION ERC grant. This funding secured me financially for four years and I also greatly appreciate your efforts to prolong the funding for as long as was possible. I had a great role model in you. Your ability to quickly tune into new complicated topics or ideas and immediately spot weak and strong points almost reminds a superpower. 

I eventually did acquire the skills to bring my vision into reality and test some thoughts of mine against the scientific method. I regained more trust in my own ideas and thinking process.

I was lucky to have great colleagues, who were important part of my academic growth: Vincent, Yuri, Xiaojing, Sourav and Vishal, with whom I led the most inspiring hours-long conversations about thermodynamics, phase field or just about life. The Friday evenings at the Chinese or Indian restaurant are the memories of togetherness, friendship and fun. Especially in the first year we were such a great study group, all so much into phase field. Xiaojing and her husband Zilong became so dear friends of mine and I cannot wait to meet their son.

I do not believe I would be able to carry my project on in the most hopeless times without Vincent Feyen. Our weekly meetings became a solid support I could lean on, a source of regular informal feedback, encouragement and an opportunity to provide the same in return. What a great paradox of life that while I was slowly gaining my self esteem and trust in my work, you became so much more critical of yours. I have always admired your intellectual courage, orientation on results and an uncompromising focus on true value in research. Thank you so much, Vincent. I wish you, Lize and Isolde all the best life can bring you. 

I am grateful for support of other wonderful people I met at MTM: Siggi Wodarz and Wouter Monnens, who helped me with the few electrodeposition experiments I did; Alina Arslanova and Vadim Trepalin, who knew it all about the struggles I was going through; Ruben Buch, with whom it was so enjoyable to prepare for the practica and who became my friend the moment we met; Yijie Liang who has always been so kind and positive.

Also, I want to express my gratitude to my Master thesis supervisor and friend from IPP František Lukáč, who gave me moral support and general advice several times. Similarly I am grateful to prof Michal Beneš from my alma mater, who is an expert on phase field method himself and despite his many obligations he proofread some of my mathematical derivations in functional calculus. Then, it was easier to use it in my research.

Especially after the fourth year of my PhD, when I got no funding and started part-time working in the automotive industry, I could strongly recognize the support from my family. With this, I deeply appreciate the support of my mum, Eva Dudrová, which enabled the part-time job and part-time PhD scheme, providing the relevant two days weekly for my PhD and sufficient funds for my family. This thesis was mostly written and compiled from texts I already had during four separate weeks of vacation I spent in Chotusice, a small village in Czechia, where my father-in-law Josef Sládek made sure I had nothing to worry about except for the thesis. I loved our lunches and talks about life. 

The same goes for Thomas Lee, who has been always as supportive and understanding as humanly possible.

The biggest thanks of all goes to my wife Alena Minar, with whom we both took a new surname after we got married because it made sense. She did everything humanly possible to support me in getting here, always by my side, and willing to accommodate the needs of the PhD into our life, uncompromising on the ambition to which we both sacrificed so much. I took a lot of the perseverance and toughness necessary to finish this from her lead. Accomplishing the PhD means for us a closure on many levels and I hope more colors will enter our life in the coming years.

\section*{Use of generative AI in this thesis}
The first complete draft of the thesis was written without any AI-generated text, but ChatGPT4 was used for brainstorming about goals of the thesis. In the following drafts, ChatGPT4 was used for improving the flow, grammar and overall comprehensibility. This applies especially to the Abstract and General Conclusion chapters. The section Goals of the thesis was rewritten by ChatGPT4 from a bullet-point list to a fluent text, but the content remained essentially equal (i.e. no ideas were added by the AI). ChatGPT was also used to aid debugging of problems in LaTeX source compilation.

The whole manuscript was reviewed by the author and he takes full responsibility.

%%%%%%%%%%%%%%%%%%%%%%%%%%%%%%%%%%%%%%%%%%%%%%%%%%
% Keep the following \cleardoublepage at the end of this file, 
% otherwise \includeonly includes empty pages.
\cleardoublepage

% vim: tw=70 nocindent expandtab foldmethod=marker foldmarker={{{}{,}{}}}
