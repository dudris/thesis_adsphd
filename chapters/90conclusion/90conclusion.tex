% !TeX root = ../../thesis.tex
\chapter{General conclusion} \label{ch_conclusion}
\section{Conclusion}
The crystallographic texture of polycrystalline deposits, whether thin or thick, is a critical characteristic, as it significantly influences many other material properties. The deposition process must be optimized for specific applications to ensure that the final product meets the necessary requirements, often by achieving a particular crystallographic texture. It is widely accepted that surface energy and its anisotropy are among the fundamental factors driving texture formation. Additionally, the generation of new grains during repeated nucleation in film deposition can markedly affect the crystallographic texture. Interface energy plays a crucial role in nucleation, and its anisotropy is expected to influence the orientation dependence of the nucleation barrier.

However, the role of interface energy anisotropy in nucleation has not been extensively studied, especially in relation to its impact on the overall evolution of the deposit. This dissertation employed various simulation methodologies to explore this issue and provide new insights.

An approach was identified to bring insight into the role of interface or surface energy anisotropy on the nucleation during polycrystalline film growth: i) develop an analytical model that relates local crystallographic orientations to the corresponding local interface energetics, which ultimately determine the stable equilibrium shape for a given orientation configuration; and ii) create a computational framework that computes the stable equilibrium shape from the input local orientations and determines the shape factor for the corresponding configuration. This computational framework would systematically explore the orientation space of the substrate and nucleus, determining the shape factor for each configuration. Ultimately, this process yields shape factor-orientation maps, which provide key insights into the anisotropy of the nucleation barrier, as conceptualized in this work.

This dissertation explored two different computational frameworks for determination of the equilibrium shapes: one based on the novel phase-field methodology (Chapter~\ref{ch_NPA_PF_methodology}) and the other based on the Winterbottom construction (Chapter~\ref{ch_paper2}). Although the phase-field methodology was found to be computationally intensive compared to the latter, it provided a solid foundation for substantial methodical development (Chapters~\ref{ch_paper1} and~\ref{ch_NPA_PF_methodology}).

The main findings from the respective chapters of this thesis are summarized below:

Chapters~\ref{ch_anisoIEintro} and~\ref{ch_introPF} provided a comprehensive literature review and discussion of relevant fundamentals.

Chapter~\ref{ch_paper1} described a multi-phase field model that incorporates the inclination dependence of interface energy. The chapter proposed two broadly applicable benchmark problems, which were used in a detailed parameter study to quantify the differences between three different model variants. Although the behavior of the model variants was largely similar, the quantitative benchmark allowed us to identify subtle differences in both the reproduction of the expected shape and the kinetics. The most significant difference between the models was observed in triple junction simulations across a wide range of configurations, revealing that only one model variant accurately reproduced the triple junction angles.

The development of good practices for determining model parameters (see Appendix~\ref{sec_suppl_param_det}), along with the publication of MATLAB tools~\cite{Minar2022dataset}, was a crucial step that allowed the delivery of reproducible results and made the model~\cite{Moelans2008} more accessible to the scientific community.
The benchmark problems developed to quantitatively assess the performance of the anisotropic curvature driving force in moving interface simulations can be used for validating any method that involves such physics: phase-field methods, level set methods, or sharp-interface methods. The MATLAB tools used for matching contours to expected shapes were also made available~\cite{Minar2022dataset}.
The governing equations for the multi-phase field model~\cite{Moelans2008} incorporating inclination-dependent interface energy were derived for all three compared model variants: IWc, IWvG, and IWvK (with IWvK and IWvG equations not previously published).
The systematic approach in the triple junction benchmark identified that only the IWvG model should be used in practical simulations involving triple junctions, as IWc and IWvK did not correctly reproduce the triple junction angles, especially in strongly anisotropic systems.

Chapter~\ref{ch_NPA_PF_methodology} explained the novel phase-field methodology for obtaining the equilibrium stable shape, given the expected contact angles and interface energy anisotropy. The \textit{domain scaling} method was then used to determine the shape factor. This methodology was applied to a case without an analytical solution (nucleus on top of a grain boundary within a substrate).

The phase-field model was enhanced by incorporating volume conservation and a boundary condition that imposed the contact angles of the particle interface on the domain boundary, which functioned as the particle-substrate interface.
The method was successfully validated on a problem involving a particle with isotropic interface energy on a plane, where the shape factor was known.
It was demonstrated that the nucleus atop a grain boundary within a planar substrate adopts an interface-energy-minimizing shape, which can be well approximated by a fifth-order Bézier polynomial.
The probability of nucleating such a shape was shown to strongly depend on the position of the grain boundary beneath the particle. When most of the particle lay on the lower-energy side of the grain boundary, the likelihood of nucleation was not significantly reduced compared to nucleation on the lower-energy side.
Due to the observed shape instability, it is suggested that atomistic simulations of nuclei on the substrate grain boundary may be more appropriate.

Chapter~\ref{ch_paper2} implemented the Winterbottom construction to determine shape factors based on local crystallographic orientations and interface energy anisotropy. Shape factor-orientation maps, which describe the anisotropy of the nucleation barrier depending on local crystallographic orientations, were generated and used as input to a custom 2D Monte Carlo algorithm for growing polycrystalline films.

Novel solutions to the generalized Winterbottom construction were developed for cases involving strongly anisotropic interface energy and arbitrary orientations.
A method was created to sample nucleus orientation when surface energy anisotropy and bottom grain orientations were known.
Monte Carlo simulations indicated that with greater anisotropy strength, nucleation rates are less dependent on the driving force but more dependent on the initial texture.
Depending on the seed texture, anisotropic surface energy could either accelerate or delay the evolution towards the interface-energy minimizing texture during nucleation.
A qualitative explanation was proposed for an unusual crystallographic texture evolution observed in very slowly electrodeposited nickel, where the course of texture evolution was suddenly altered by nucleation despite the low growth rate.

The main objectives of this thesis were: to develop a phase-field based methodology capable of determining nucleation barrier anisotropy, and to gain insight into the potential role of interface energy anisotropy on repeated nucleation in growing polycrystals and its impact on crystallographic texture. The key results summarized above from Chapter~\ref{ch_NPA_PF_methodology} and Chapter~\ref{ch_paper2} clearly demonstrate that these primary objectives were successfully achieved.


\section{Outlook}

The following potential future research directions are proposed based on the findings of this thesis.

\textbf{Improvements to the Multi-Phase Field Model:} The multi-phase field model used in this work~\cite{Moelans2008, Minar2022} (specifically, the multi-continuum field model) is capable of reproducing the triple junction angles between distinct isotropic interfaces. However, for systems with strong anisotropy, the model suffers from variations in interface width. Developing a method to maintain a constant interface width while preserving thermodynamic consistency would enhance the utility and competitiveness of the model, particularly in comparison with other multi-phase field models~\cite{Steinbach1999, Nestler2005} that do not encounter this issue in systems with distinct isotropic interfaces.

\textbf{Benchmark Problem for Triple Junctions with Inclination-Dependent Interface Energies:} Another challenge related to multi-phase field models is simulating triple junctions of interfaces with inclination-dependent interface energy. While the most straightforward application of these models is grain growth—which in reality involves strong inclination dependence of grain boundary energy—no reliable phase field simulation of truly anisotropic grain growth has yet been conducted. Problems with ghost phases distorting the triple junction angles, as well as varying interface widths, persist even in simpler pair-distinct isotropic systems. Some models have been more successful than others in mitigating these issues, but the inclination dependence of interface energy at triple junctions adds an additional layer of complexity that requires optimization. The first step in this direction should be the development of a dedicated benchmark problem to quantify the accuracy of current models and identify potential weaknesses in their formulations.

\textbf{Phase-Field Simulations of Wetting on Curved Surfaces with Control Over Inclination Angle:} In this work, the numerical solution to the phase-field governing equations was obtained using the finite difference method, typically applied on rectangular domains. In simulations related to \textit{domain scaling}, the bottom domain boundary represented the particle-substrate interface. By controlling its shape, further insights could be gained into equilibrium shapes on curved surfaces, where the Winterbottom construction cannot be used. Implementing the finite element method with adaptive meshing could open new avenues for research in this area. 

\textbf{Developing More Sophisticated Models of Particle-Substrate Interactions:} In the wetting simulations with controlled interface inclination angle at the domain boundary, an assumption is made regarding how the particle-liquid interface interacts with the substrate, and this determines the inclination angles that should be imposed on the interface in contact with the substrate. In this work, the contact angle was modeled as a step function with two constants along the bottom domain boundary, and the stable equilibrium shape was sought. However, more complex scenarios involving different spatial and/or temporal dependencies of the controlled inclination angle along the domain boundary are possible. These variations of the imposed contact angle may represent local variations of the substrate surface properties (e.g. on a substrate with some periodic pattern or very fine grain structure). 

\textbf{Simulation of Particle Dynamics:} The phase-field simulations in this work were limited to static configurations, focusing on equilibrium shapes. Incorporating dynamic phenomena such as fluid flow or additional thermodynamic fields like concentration or temperature could provide further insights into how particles grow and/or interact with their environment over time. This would open up new possibilities for studying time-dependent processes, such as solid-state wetting, reactive liquid wetting or growth in various scenarios like deposition or solidification. It is possible to emulate change of the substrate-particle interactions on the course of simulation by temporal variation of the imposed contact angle of the solid-liquid particle interface at the domain boundary. 
	
\textbf{Application of Nucleus Orientation Sampling in Other Polycrystalline Growth Simulations:} The nucleus orientation sampling algorithm developed here could be applied to more comprehensive models of polycrystalline growth. The algorithm is independent on the simulation method - it samples nucleus orientation based on the substrate orientation and the interface energies and their anisotropy. These simulations have the potential to bring further insights more applied to the specific context (e.g. solidification, chemical vapor deposition, electrodeposition).

\textbf{Extension of Shape Factor-Orientation Maps:} The shape factor-orientation maps developed in this thesis could be expanded to incorporate more complex anisotropy functions, such as three-dimensional anisotropy or variations in other bulk fields (e.g. adsorbent concentration or temperature). These extensions would add more independent variables to the shape factor-orientation maps, hence another dimension. Including such factors would make the processing more challenging but also would allow the maps to offer a more complete picture of nucleation barrier anisotropy and extend their applicability to a wider range of materials and systems. Another possible extension is to assume that the particle-substrate interface is inclination-dependent and determine the equilibrium shapes using the double Winterbottom construction, which takes anisotropy of this interface into account as well. 



%%%%%%%%%%%%%%%%%%%%%%%%%%%%%%%%%%%%%%%%%%%%%%%%%%
% Keep the following \cleardoublepage at the end of this file, 
% otherwise \includeonly includes empty pages.
\cleardoublepage

% vim: tw=70 nocindent expandtab foldmethod=marker foldmarker={{{}{,}{}}}
