% !TeX root = ../../thesis.tex
\chapter{General conclusion} \label{ch_conclusion}
\section{Conclusion}
Crystallographic texture of polycrystalline deposits (both thin or thick) is one of their most important characteristics, because it determines many other deposit properties. The deposition process must be optimized for the particular application so that the product meets the requirements, which usually corresponds to achieving a particular crystallographic texture. There is a wide consensus that the surface energy and its anisotropy is one of fundamental texture forming factors. At the same time, it is known that new grains may be generated during film deposition by repeated nucleation and this process certainly may affect the crystallographic texture. Interface energy plays key role in nucleation and it is thus natural to expect that when it is anisotropic, it would have some impact on the orientation dependence of the nucleation barrier as well. 

However, the role of interface energy anisotropy in nucleation has not received much attention and certainly not in conjunction with the concurrent effect on the deposit evolution. This dissertation used various simulation approaches to bring some insight in this matter.

It was identified, that one of possible approaches to bring insight into the role of interface or surface energy anisotropy on the nucleation during polycrystalline film growth was to develop i) an analytic model relating the local crystallographic orientations to the local interface energetics, which eventually determine the stable equilibrium shape for that orientations configuration and ii) some "solver" which would provide the stable equilibrium shape from the input local orientations and then determined the shape factor for the shape. The space of substrate and nucleus orientations was to be probed by the "solver", determining the shape factor every time and eventually to provide the shape factor-orientation maps, which are the essence of the nucleation barrier anisotropy as this work puts it.

This dissertation investigated two concepts of the "solver": one based on the novel phase-field methodology (Chapter~\ref{ch_NPA_PF_methodology}) and one utilizing the Winterbottom construction (Chapter~\ref{ch_paper2}). Even though the phase field methodology turned out to be overly computationally intensive in comparison with the latter, great amount of methodical work was done on the phase field model (Chapters~\ref{ch_paper1} and~\ref{ch_NPA_PF_methodology}). 

%Short summary of each chapter can be found in the section~\ref{sec_Thesis_overview}, hence it will not be repeated here, but the main results obtained in the respective chapters of this thesis will be listed instead.
The main results obtained in the respective chapters of this thesis were listed below.

\textbf{The Chapters~\ref{ch_anisoIEintro} and~\ref{ch_introPF}} discuss the literature review and relevant fundamentals.

\textbf{Chapter~\ref{ch_paper1}} described a multi-phase field model including inclination dependence of interface energy. Then it proposed two generally applicable benchmark problems, which were used in a thorough parameter study to quantify the differences between three different model variants. Even though the model variants behavior was very similar, the quantitative nature of the benchmark allowed to spot the minor differences in both how they reproduced the expected shape as well as in terms of the kinetics. The most relevant difference between the models was revealed after triple junction simulations in a wide range of configurations, which showed that only one model variant correctly reproduced the triple junction angles.
\begin{itemize}
	\item The developed good practices in model parameters determination (see Appendix~\ref{sec_suppl_param_det}) together with the published MATLAB tools~\cite{Minar2022dataset} were in fact a crucial step, which i) allowed delivery of reproducible results and ii) made the model~\cite{Moelans2008} more accessible to the scientific community.
	\item The developed benchmark problems to quantitatively assess performance of the anisotropic curvature driving force in a moving interface simulation can be used for validation of any method which involves such physics: phase-field method, level set method or sharp-interface methods. The MATLAB tools used for the contours matching to the expected shapes were made available as well~\cite{Minar2022dataset}.
	\item The governing equations for the multi-phase field model~\cite{Moelans2008} including the inclination-dependent interface energy were derived for all the three compared model variants IWc, IWvG and IWvK (for IWvK and IWvG they were not published before).
	\item The systematic approach in the triple junctions benchmark identified that only the IWvG model should be used in practical simulations involving triple junctions, because IWc and IWvK did not reproduce the triple junctions correctly, especially for strongly anisotropic systems.
\end{itemize}

\textbf{Chapter~\ref{ch_NPA_PF_methodology}} explains the novel phase field methodology for obtaining the equilibrium stable shape, given the expected contact angles and interface energy anisotropy. The \textit{domain scaling} method was then used to determine the shape factor. The methodology was applied to a case without analytic solution (nucleus on top of a grain boundary within a substrate).
\begin{itemize}
%	\item The concept of phase field simulation able to deliver a stable shape of a particle with anisotropic interface energy on a planar substrate was delivered. The concept required two phase fields in a rectangular domain, volume conservation of the particle phase and also control over the inclination angle of the interface at the domain boundary, which acts as the particle-substrate interface.
	\item The phase field model and its implementation was extended by the volume conservation and a boundary condition imposing the contact angles of the particle interface on the domain boundary, which then acted as the particle-substrate interface.
	\item The method was successfully validated on a problem with a particle with isotropic interface energy on a plane, where the shape factor was known.
	\item It was shown that the nucleus on top of a grain boundary within a planar substrate has an interface-energy-minimizing shape which can be well approximated by Bézier polynomial of order 5.
	\item It was shown that the probability of such shape nucleation strongly depends on the position of the grain boundary below the particle. When most of the particle laid on the lower-energy side of the grain boundary, nucleation of the shape was not much less likely than nucleation on the lower-energy side.
	\item Due to the obtained shape instability, it is reasoned that atomistic simulations of nuclei on the substrate grain boundary would be more appropriate.
\end{itemize}

\textbf{Chapter~\ref{ch_paper2}} implemented the Winterbottom construction to deliver shape factors given the local crystallographic orientations and interface energy anisotropy. The shape factor-orientation maps describing anisotropy of the nucleation barrier depending on local crystallogrphic orientations were obtained and used as input to a custom 2D Monte-Carlo algorithm of growing polycrystalline film.
\begin{itemize}
	\item Novel solutions to the generalized Winterbottom construction were described in cases with strongly anisotropic interface energy and arbitrary orientations.
	\item A method was developed, which allowed sampling of nucleus orientation when the surface energy anisotropy and bottom grain orientations were known.
	\item Monte Carlo simulations: with larger strength of anisotropy, the nucleation rate is less dependent on the driving force, but more dependent on the initial texture.
	\item Monte Carlo simulations: depending on the seed texture, the nucleation with anisotropic surface energy may both hasten the evolution towards the interface-energy minimizing texture or retard it.
	\item Proposed qualitative explanation to a peculiar experimental crystallographic texture evolution in very slowly electrodeposited nickel. There, the course of texture evolution was suddenly changed by nucleation (despite the low growth rate).
\end{itemize}

The main goals of this thesis were: to develop a phase-field based methodology capable of determination of nucleation barrier anisotropy. Then, an insight was sought into the possible role of interface energy anisotropy on the repeated nucleation in the growing polycrystals and the impact of this on crystallographic texture. The above summary of key results produced in Chapter~\ref{ch_NPA_PF_methodology} and Chapter~\ref{ch_paper2} clearly show that the main goals of the thesis were met.
	
\section{Outlook}
Possible future directions for the research carried out in this thesis were outlined below.

\textbf{The multi-phase field model improvements} The multi-phase field model used in this paper~\cite{Moelans2008, Minar2022} (more specifically multi-continuum field model) is able to reproduce the triple junction angles between distinct isotropic interfaces, however for strong anisotropy it suffers from the variation in interface width. Finding a way to assure constant interface width while preserving the thermodynamic consistency would make the model more useful and also more competitive, because there are multi-phase field models~\cite{Steinbach1999, Nestler2005} which do not have this difficulty in the systems with distinct isotropic interfaces.

\textbf{Benchmark problem for triple junctions with inclination-dependent interface energies} Another issue related to multi-phase field models in general are triple junctions of interfaces with inclination-dependent interface energy. Even though the most straightforward application of multi-phase field models is grain growth (which in practice exhibits strong inclination dependence of grain boundary energy), no reliable simulation of truly anisotropic grain growth has been carried out yet. This is because already with the much simpler pair-distinct isotropic systems there are problems with ghost phases distorting the triple junction angles and possibly there is the varying interface width. Some models were more successful than others in mitigating these problems but still the inclination dependence of interface energy in triple junctions represents additional level of complexity the models must be optimized for. The first natural step in this line of development is a dedicated benchmark problem, which would allow to quantify the accuracy of the present models and possibly would help with identification of weak spots in the model formulations. 

\textbf{Phase field simulations of wetting on curved surfaces with the control over inclination angle} In this work, the numeric solution to the phase field governing equations was delivered by finite difference method, which is usually carried out on rectangular domains. In the simulations related to \textit{domain scaling}, the bottom domain boundary represented the particle-substrate interface. By controlling its shape, one might gain some further insight into the equilibrium shapes on curved surfaces. That is also the case for which the solution is unknown. Finite elements method used with adaptive meshing could open new possibilities in this direction of research.
	
\textbf{Develop more sophisticated models of particle-substrate interactions to provide further insights} In the wetting simulations with controlled interface inclination angle at the domain boundary, one must assume how exactly the particle-liquid interface interacts with the substrate and have a way to derive the inclination angles which should be imposed on the interface in contact with the substrate. Here, the contact angle was a step function with two constants along the bottom domain boundary and the stable equilibrium shape was seeked. However, more complicated scenarios involving different spatial and/or temporal dependence of the controlled inclination angle along the domain boundary are possible and insights may be obtained from having a meaningful model relating the particle-substrate interactions in a particular point on the domain boundary to the imposed local inclination angle.
	
\textbf{Simulate particle dynamics} In the presented work, no dynamics was simulated in the phase-field simulations. The equilibrium stable shape is a snapshot of an evolving material. It is relevant in nucleation, but the simulation did not capture anything what happened next to the particle in interaction with its surroundings. Depending on the context, addition of further physics would certainly open many new possibilities in these wetting simulations. E.g. for the liquid wetting simulations the fluid flow ought to be added. On the other hand, solid-state wetting simulations could readily be simulated with this model. 
	
\textbf{Use the nucleation sampling algorithm in a more sophisticated simulation of a growing polycrystal} The nucleus orientation sampling developed here could be used in a more sophisticated model of growing polycrystal to bring further insights. 

\textbf{Extension of the shape factor-orientation maps} The shape factor maps could be constructed for 3D anisotropy function or some local values of relevant fields (like e.g. adsorbent concentration $c$) may be included in the analytic model relating the local crystallographic orientations to the wetting parameter $\Gamma(\alpha_1,\alpha_2, c)$. That would add another dimension to the shape factor-orientation map. Additionally, the shape factor orientation maps could be constructed for another anisotropy functions.



%%%%%%%%%%%%%%%%%%%%%%%%%%%%%%%%%%%%%%%%%%%%%%%%%%
% Keep the following \cleardoublepage at the end of this file, 
% otherwise \includeonly includes empty pages.
\cleardoublepage

% vim: tw=70 nocindent expandtab foldmethod=marker foldmarker={{{}{,}{}}}
