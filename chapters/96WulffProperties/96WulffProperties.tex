% !TeX root = ../../thesis.tex
\chapter{Selected Wulff shape properties} \label{ch_appendix_wulff_properties}
\section{Selected geometric properties of Wulff shape}\label{sec_appendix_anisofun_props}
With $h(\theta)=1+\delta\cos(n\theta)$, the minimal distance $R_{min}$ between the Wulff shape center and the contour can be related to $R_W$ as
\begin{equation} \label{eq_appdx_wulff_minradius}
	R_{min} = R_W(1-\delta) \,,
\end{equation}
which holds for arbitrarily strong anisotropy because the minimal-radius point normal is always inclined under a non-missing angle. This formula was used to find the radius $R_W$ of the phase field contour. Then, the phase field contour was scaled to unit radius and compared to the analytic Wulff shape (by means of Hausdorff distance, see the next subsection). 

The radius of curvature $\varrho(\theta)$ of Wulff shape is locally
\begin{equation} \label{eq_Wulff_radius_curvature}
	\varrho(\theta) = R_W[h''(\theta) + h(\theta)] \,,
\end{equation}
which may technically get negative for some $\theta$. Because that does not have any geometric interpretation, the Wulff shape does not contain such oriented segments and the angles $\theta_m$ fulfilling $\varrho(\theta_m)\leq0$ are called missing or forbidden. \\

For a given radius $R_W$, the area of a Wulff shape $A_W$ differs from the area of a circle $A=\pi R_W^2$ by the anisotropy-dependent factor $C_W(\Omega,n) = A_W/A$. For weak anisotropy $\Omega<1$ there is an analytic expression 
\begin{equation}
	C_W = 1-\frac{\Omega^2}{n^2-1} \,,
\end{equation}
but there is none for strong anisotropy $\Omega>1$. The factor $C_W(\Omega,n)$ was obtained numerically for $n=4$ in the range $0\leq\Omega\leq7.5$ (the Wulff shape was plotted, its area numerically computed and divided by $\pi R^2$). There, the $\Omega$-dependence was fitted very well  by a polynomial of 4-th order $C_W(\Omega,4)= \sum_{i=0}^4a_i\Omega^{N-i}$ with $a_0=-0.00032 ,a_1=0.00639 ,a_2=-0.04219 , a_3=0.00034 , a_4=1.00000$. 

\section{Shrinkage rate of a Wulff shape} \label{sec_appendix_wulff_shrrate}
The normal velocity of anisotropic curvature-driven interface in 2D is~\cite{Abdeljawad2018}
\begin{equation}
	v_n(\theta,t)=-\frac{\mu}{\varrho(\theta,t)}\sigma_0[h''(\theta) + h(\theta)] 
\end{equation}
where $\mu$ is the interface mobility and $\sigma_0[h''(\theta) + h(\theta)]$ is interface stiffness. When it is the
Wulff shape shrinking, the radius of curvature $\varrho(\theta,t)$ can be expressed as in equation~\eqref{eq_Wulff_radius_curvature} and the inclination-dependent part of the interface stiffness is cancelled out. The normal velocity is then constant along the perimeter
\begin{equation} \label{eq_wulff_norm_velocity}
	v_n(t) = - \frac{\mu \sigma_0}{R_W(t)} \,.
\end{equation}
The sign is such that the interface moves towards the center of curvature.

Because the area of a Wulff shape is $A_W=\pi R_W^2C_W(\Omega,n)$, the shrinkage rate of a Wulff shape is
\begin{equation}\label{eq_wulff_shrrate_deriv}
	\begin{split}
		\frac{\mathrm{d}A_W}{\mathrm{d}t} &= \frac{\mathrm{d}}{\mathrm{d}t}[\pi R_W^2(t)C_W(\Omega,n)]    \\
		&= 2\pi C_W(\Omega,n) R_W(t) \frac{\mathrm{d}R_W}{\mathrm{d}t} \,.
	\end{split}
\end{equation}
Because  the normal velocity is constant along the perimeter, the point $R_{min}$ from equation~\eqref{eq_appdx_wulff_minradius} moves also by $v_n(t)$. Importantly, the normal of interface in $R_{min}$ is aligned with the radial direction of the polar coordinates, hence we can write $v_n=\mathrm{d}R_{min}/\mathrm{d}t$. From equation~\eqref{eq_appdx_wulff_minradius} then must be
\begin{equation}
	\frac{\mathrm{d}R_{W}}{\mathrm{d}t}=\frac{1}{1-\delta}\frac{\mathrm{d}R_{min}}{\mathrm{d}t}=\frac{1}{1-\delta}v_n(t) \,,
\end{equation}
which together with~\ref{eq_wulff_norm_velocity} implies in \ref{eq_wulff_shrrate_deriv}
\begin{equation}
	\frac{\mathrm{d}A_W}{\mathrm{d}t} = 2\pi\mu\sigma_0  \frac{C_W(\Omega,n)}{1-\delta} 
\end{equation}

Taylor and Cahn~\cite{Taylor1998} proved that the shrinkage rate for the equilibrium shape must be a constant, which is met by the final expression.



%%%%%%%%%%%%%%%%%%%%%%%%%%%%%%%%%%%%%%%%%%%%%%%%%%
% Keep the following \cleardoublepage at the end of this file, 
% otherwise \includeonly includes empty pages.
\cleardoublepage

% vim: tw=70 nocindent expandtab foldmethod=marker foldmarker={{{}{,}{}}}
