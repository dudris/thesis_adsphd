% !TeX root = ../../thesis.tex
\chapter{Literature review}\label{ch_anisoIEintro}

\section{Microstructure formation in coatings}
Microstructure here denotes both the morphology on the surface, shape and size of grains and their crystallographic texture.

The experimental investigations of the texture-forming factors in deposits are usually specific for the particular deposition method, like physical vapour deposition, sputtering or electrodeposition. Despite some obvious differences between the methods, it is natural to expect the textures to be formed by the same mechanisms. These are necessarily altered by the method-specific deposition conditions, thus giving rise to some behaviors which are then considered as standard for that type of deposition. 

For example, the Structure Zone Model (SZM)~\cite{Barna1998} is often used to describe the dependence of microstructure on the relative deposition temperature in PVD and sputtering methods, but not so in electrodeposition. There, rather the Winand diagram~\cite{Winand1992} is used instead, having the axes representing degree of chemical surface inhibition and current density, i.e. deposition rate. Both of these can be perceived as state diagrams, where the individual regions qualitatively indicate the one microstructure/morphology which is most likely to be stable for those deposition conditions. Interestingly, in electrodeposition the crystallographic texture of the coating does not necessarily imply a particular deposit morphology in electrodeposition

Barna~\cite{Barna1998} claims, that the different crystallographic textures observed in PVD or sputter deposition can be correlated to the respective structure zone. Specifically, he uses the terms \textit{competitive growth texture} and \textit{restructuration growth texture}, which correspond to the ones where the resulting texture depends on either the anisotropy in growth rate or on the concurrent minimization of surface and grain boundary energy, respectively.

In electrodeposition, the situation is more complicated than in PVD or sputtering by the presence of the double layer on the metal surface and the complex interplay of the electric field and chemical reactions between all the species.  Nevertheless, experimental work largely shows that, as far as three-dimensional nucleation is concerned, the same observations are made in electrocrystallization
and in physical crystallization~\cite{Winand1992}: increasing the deposition rate and/or increasing the inhibition usually results in increased nucleation rate and thus eventually in the reduction in the deposit grain size. .

The inhibition refers to some specific interactions between the deposit and the parent phase. The inhibiting species attaches to low-energy sites where the growth might easily proceed and then the adatoms must find another spot to settle down, increasing thus the nucleation probability~\cite{Winand1992}. The inhibitor may be a natural constituent of the electrolyte in electrodeposition or simply some impurity. In PVD deposition of aluminium with variable amounts of oxygen, it was observed that increased oxygen amounts reduced the grain size up to nanocrystalline or even amorphous coating of alumina~\cite{Barna1998}.

A separate field of research involves stress in the deposits~\cite{Thornton1989, Thompson1993, Chason2015, Abadias2018}. Both tensile and compressive stress are observed in the deposits depending on the deposition conditions and deposited material. Generally speaking, higher temperatures and lower deposition rates (i.e. deposition closer to thermodynamic equilibrium) lead to behavior, where a compressive stress develops with increasing thickness of the film, but after the deposition stops, the stress either completely or partly relaxes. This is highly dependent on the material too, where materials with higher self-diffusion coefficient (lower melting point) tend to exhibit such behavior~\cite{Chason2002}. On the other hand, high growth rates and low deposition temperatures (i.e. rather off-equilibrium deposition) lead to tensile stresses, which do not relax after the deposition termination. Materials with lower self-diffusion coefficients (higher melting points) are associated with these. Another source of stress can be interactions with the substrate.

Stress can suppress or promote grain growth~\cite{Thompson1993}. It is then natural to consider stress relaxation as another texture-forming process. As was mentioned in the Introduction, the fundamental texture-forming quantities are interfacial energy, surface diffusivity of adatoms, grain boundary mobility and grain boundary and lattice diffusion~\cite{Szpunar1997, Suwas2014}. However, it is the particular state of the material and the deposition conditions, which then drives the deposit to evolve in particular ways.

Thompson~\cite{Thompson1993} developed an abstracted model of columnar microstructure and expressed its total strain and total interface energy as function of thickness. Because the thickness dependence of these energies was different, the model suggested that the texture could be dominated by minimization of interface energy below certain (temperature-dependent) thickness and above it, the texture would be dominated by the strain energy minimization. Especially in the light of the previous paragraph, it is clear that the stress state of a real deposit does not depend only on the deposit thickness. Nevertheless, the idea of having different mechanisms competing over the course of texture/morphology evolution as the film thickness increases is strongly based in experimental observation. Also, the terms \textit{interface-energy minimization texture} or \textit{strain-energy minimization texture} are used in practice~\cite{Alimadadi2016}.

Specifically in electrodeposition, the inhibition in relation to the morphology and texture was studied extensively~\cite{Winand1992}. Amblard~\cite{Amblard1979} determined the particular species responsible for the texture variation in nickel electrodeposits from Watts electrolyte when varying the pH, current density and adding separately a brightening and levelling agents. These species selectively inhibited the metal surface and thus introduced different textures. A more recent study~\cite{BergenstofNielsen1997} reviewed other hypotheses about the possible texture-forming processes in nickel electrodeposition from Watts electrolyte and after transmssion electron microscopy study of one deposit, they concluded that the inhibition was the most likely scenario. 

The same group later proposed~\cite{Rasmussen2001}, that the texture of an electrodeposit could develop from a zone mainly affected by the substrate (epitaxy, chemical reactions), through a zone of mixed control to a zone affected solely by the deposition conditions (selective inhibition, deposition rate). However, in a more recent study~\cite{Alimadadi2016} involving microtexture measurement through thickness of the nickel electrodeposits obtained at different pH and current density, the authors could not confirm the existence of the transition zone as a simple mixture of the substrate- and conditions-affected zones. 

\section{Anisotropy in interface energy vs. kinetic coefficient}
A fundamental work by Damjanovic~\cite{Damjanovic1966} reports the morphology of copper depositions on different Cu monocrystalline substrates. It also states, that the exchange current densities $i_0$ for Cu electrodeposition on the different crystallographic planes (i.e. kinetic coefficient in the deposition/dissolution reaction) are in the following sequence of magnitude:
\begin{equation}
	i_0^{(111)}<i_0^{(001)}<i_0^{(011)} \,.
\end{equation}
Interestingly, the same order comes for the \textit{inteface energies} of the respective planes in vacuum for both Cu and Ag based both on broken bond model by Wang~\cite{Wang2000} and DFT calculations by Tran~\cite{Tran2016}. 

In addition to that, it is known that the fundamental crystal forms obtained assuming a single anisotropy in both interface energy and kinetic coefficient (i.e. growth rate) are geometrically similar~\cite{Kobayashi2001,Salvalaglio2015}. More specifically, if the total interface energy of an isolated crystal is minimized with a constant volume, the resulting so-called Wulff shape depends on the interface energy anisotropy. This Wulff shape is geometrically similar to the shape obtained by growth of a spherical crystal under the effect of anisotropic growth rate with the same anisotropy as the interface energy had. This idea is illustrated in Figure~\ref{fig_ECS_vs_KCS_wulff_construction}.

\begin{figure}
	\centering
	\includegraphics[width=0.6\textwidth]{ECS_vs_KCS_wulff_construction.png}
	\caption{Illustration of geometric similarity between Equilibrium Crystal Shape (ECS) driven by anisotropy in interface energy and Kinetic Crystal Shape (KCS) obtained by growth with growth rate having the same anisotropy. The top polar plot is the anisotropy function $h(\theta)$. Source:~\cite{Salvalaglio2015}.}
	\label{fig_ECS_vs_KCS_wulff_construction}
\end{figure}

It is not surprising to have the anisotropies in the interface energy and kinetic coefficients somehow related, because they both must derive from the state of the same interface. However, it is certainly interesting to realize that despite them being substantially different physical quantities, used for description of different processes, their anisotropy may manifest in similar ways. 

This was supported also by simulations of 2D growth of polycrystalline zeolite film using multi-phase field method~\cite{Wendler2011}. Reportedly, the same microstructures were obtained when assuming the same anisotropy in either interface energy or in the kinetic coefficient. 

While in crystal growth, the anisotropy in interface energy and kinetic coefficient may be interchangeable to some extent in terms of effect on the resulting shape (not on the shape evolution, though), there are necessarily processes where there is no overlap due to different nature of the quantities. For example, in nucleation, the nucleation barrier depends on the interface energy but does not depend on the kinetics. 

This interchangeability makes the experimental determination of these properties in-situ during deposition very complicated. However, in models of the deposition and in simulations, it is in fact necessary to distinguish the two quantities and then it is possible to distinguish also their impact of their anisotropy on the process. The mentioned phase-field method is one of methods capable to implement these quantities independently.

In the light of previous paragraphs, it is clear that while in some simulations it may not be a big problem to neglect anisotropy in either kinetic coefficient or interface energy as long as it is preserved in the other quantity. However, in the processes where these quantities are not interchangeable (e.g. anisotropic interface energy in nucleation barrier), neglecting the implications of anisotropy results in incomplete representation of the process.

\section{Nucleation with anisotropic interface energy}


%\section{Capillary vector formalism}
%Let the interface energy $\sigma$ depend on the local orientation of the interface described by its unit normal $\bm{n}$, i.e. $\sigma=\sigma(\bm{n})=\sigma_0 f(\bm{n})$. The function $f(\bm{n})$ is called anisotropy function and $\sigma_0$  is a scaling parameter with dimensions of interface energy. Formal requirements on the anisotropy function can be found e.g. in~\cite{Kobayashi2001}. A very useful formalism for thermodynamic description of these anisotropic interfaces is so-called capillary vector $\bm{\xi}$ (a.k.a. Cahn-Hoffmann vector or $\xi$-vector), defined as 
%\begin{equation}
%     \bm{\xi} = \mathrm{grad}(r\sigma(\bm{n}))\,,
% \end{equation}
%where $r$ is radial distance from origin, which can be also seen as magnitude of scaled normal vector $\bm{r}=r\bm{n}$. Level sets of the scalar function $r\sigma(\bm{n})$ are geometrically similar to the anisotropy function $f(\bm{n})$, which is continuously scaled by the radius $r$.
%
%Capillary vector was introduced and investigated by Hoffman and Cahn in~\cite{Hoffman1972,Cahn1974}, who showed that this formalism is consistent with major laws governing behavior of (anisotropic) interfaces like Gibbs-Thompson equation (relating chemical potential and isotropic surface curvature), Herring's equation (chemical potential on anisotropic curved surface), Wulff shape construction, force balance at multi-junctions and others. 
%
%A defining property of capillary vector is that its component normal to the surface $\xi_\perp$ is in magnitude equal to the local value of the interface energy and the surface-tangent component points in the direction of steepest change in interface energy and is proportional to the change. More specifically  
%\begin{align}
%     \xi_\perp &= \bm{\xi}\cdot\bm{n}=\sigma(\bm{n}) \\
%     \xi_\| &= \left(\frac{\partial \sigma}{\partial \theta}\right)_{max} \,,
% \end{align}
%where $\theta$ is the angle representing change in orientation of $\bm{n}$ in the direction of maximal angular rate change. The vector can thus be written
%\begin{equation}
%     \bm{\xi}(\bm{n}) = \sigma(\bm{n})\bm{n} + \left(\frac{\partial \sigma}{\partial \theta}\right)_{max} \bm{t} \,,
% \end{equation}
%with $\bm{t}$ being unit vector tangent to the surface pointing in the described direction. Useful is expression of capillary vector in spherical coordinates 
%\begin{equation}
%     \bm{\xi}(\theta,\phi) = \sigma(\theta,\phi)\bm{\hat{r}} + \left(\frac{\partial \sigma}{\partial \theta}\right)\bm{\hat{\theta}} + \frac{1}{\sin(\theta)}\left(\frac{\partial \sigma}{\partial \phi}\right)\bm{\hat{\phi}} \,,
% \end{equation}
%which is only function of the polar and azimuthal angle $\theta,\phi$, respectively. In 2D in polar coordinates the normal and tangent vector can be written as functions of the interface normal angle $\theta$ as
%\begin{equation} \label{eq_xivec_2D}
%     \bm{\xi}(\theta) = \sigma(\theta)\begin{bmatrix}
%           \cos(\theta)  \\
%           \sin(\theta) 
%      \end{bmatrix} + \left(\frac{\partial \sigma}{\partial \theta}\right)\begin{bmatrix}
%           -\sin(\theta) \\
%           \cos(\theta)  
%      \end{bmatrix} \,.
% \end{equation}
%
%\section{Interface stiffness}
%
%\section{Wulff shape}
%Wulff shape is such shape, which minimizes the interface energy given a constant volume. In 2D it can be obtained as the inner convex hull of all tangent straight lines to the anisotropy function $h(\theta)$. This can be used to define the Wulff shape $W$ as a set of points $\bm{x}$
%\begin{equation}
%     W = \{ \bm{x}\in\mathbb{R}| \bm{x}\cdot\bm{n}\leq \sigma(\bm{n})\} \,,
% \end{equation}
%which is a geometrical representation of what was said above - projection of the position vector $\bm{x}$ of a point within the Wulff shape to the interface normal vector $\bm{n}$ must be smaller than the anisotropic interface energy. This defines the inner hull of the tangent lines.
%
%When the heads of capillary vectors for all interface normal angles $\theta$ are connected (having a common starting point in the origin), so-called $\xi$-plot is obtained. It turns out, that the $\xi$-plot coincides with the Wulff shape~\cite{Hoffman1972}. Hence, the expression~\eqref{eq_xivec_2D} can be used to plot the 2D Wulff shape in cartesian coordinates. This is entirely true as long as there are no corners or facets on the Wulff shape, in which cases the use of capillary vecor for Wulff shape plotting must be appropriately modified (see below for details).
%
%In~\cite{Cahn1974} the authors re-state he problem of finding equilibrium shapes of crystals with anisotropic interfaces under general conditions. The shape is found as a solution to equation
%\begin{equation} \label{eq_xivec_equilibrium_shape}
%     \Delta\Omega_v + \nabla_S\cdot \bm{\xi} = 0 \,,
% \end{equation}
%where $\Delta\Omega_v$ is the bulk driving force, specifically the difference in grand-canonical potential of the two phases, and $\nabla_S\cdot$ is surface divergence. The surface $\bm{r}$ is seeked, complying with the above equation. A simple trick is used to find the shape of an isolated particle. Surface divergence over the surface itself is simply $\nabla_S\cdot\bm{r}=2$, hence we can write
%\begin{align}
%     1 = \nabla_S\cdot\bm{r}/2 &= -\nabla_S\cdot\bm{\xi}/\Delta\Omega_v \\
%      \nabla_S\cdot(\bm{r} +2\bm{\xi}/\Delta\Omega_v) &=0 \,,
% \end{align}
%leading to the result
%\begin{equation}
%     \bm{r} = -\frac{2}{\Delta\Omega_v}\bm{\xi} \,.
% \end{equation}
%
%Two different types of singularities may occur in Wulff shapes, which must be treated in a specific way: corners and facets. 
%
%Corners represent a discontinuity of normal angle along the Wulff shape perimeter. They occur when the anisotropy of the interface energy is such, that interface stiffness gets negative for some normal angles $\theta_f$. These angles are called forbidden and they are thermodynamically unstable. For this reason, they are not present on the Wulff shape. Additionally, there are normal angles $\theta$ which are thermmodynamically stable but are not present on the Wulff shape. The latter ones correspond to angles, where the inverse interface energy $1/\sigma(\theta)$ is non-convex but still have positive interface stiffness. All forbidden angles fall into this non-convex region. By replacing the non-convex part of this function by straight lines a regularized inverse anisotropy function $1/\bar{\sigma}(\theta)$ is obtained. Wulff shape corresponding to the regularized anisotropy function $\bar{\sigma}(\theta)$ does not contain the "ears" behind the corners. The interface stiffness of both the allowed and forbidden angles which form the "ears" were thus set to zero.
%
%The facets were not treated in this work, hence they are not further discussed.
%
%\section{Force balance in triple junction and stability of its configuration} \label{sec_intro_trijun_forcebalance_stability}
%A triple junction in 2D is simply a point and force balance is achieved if the capillary vectors of the three interfaces sum to zero, i.e. 
%\begin{equation} \label{eq_trijun_forcebalance_sum_xivec}
%     \bm{\xi}_{12}+\bm{\xi}_{23}+\bm{\xi}_{13}=0
% \end{equation}
%The problem of force balance in triple junctions was treated in detail in~\cite{Marks2012}. Even though the capillary vector formalism was not used, a condition equivalent to~\eqref{eq_trijun_forcebalance_sum_xivec} was obtained from geometrical arguments (note that the 90-degree rotation does not change meaning of the equation)
%\begin{equation} \label{eq_force_balance_3jun_aniso}
%     \sum_{i=1}^3 \left[ \sigma_i(\vartheta_i)\hat{\bm{t}}_i +\frac{\mathrm{d} \sigma_i}{\mathrm{d} \vartheta_i}\hat{\bm{n}}_i \right] = 0 \,,
% \end{equation}
%where the subscripts now denote interfaces numbered from 1 to 3. $\vartheta_1,\vartheta_2,\vartheta_3$ are angles under which the tangent of each interface is oriented (all w.r.t.\ x axis),  the anisotorpic interface energies are $\sigma_i(\vartheta_i)$. The triple junction configuration is defined by the triplet of the angles $\vartheta_i$. However, it was pointed out, that the above condition is only necessary for the triple junction configuration to be stable, i.e. it does not guarantee the configuration stability. Validity of~\eqref{eq_force_balance_3jun_aniso} assures a stationary point in Gibbs free energy, but in order to have a stable configuration, the energy must be in local minimum. To confirm that, second derivatives of Gibbs free energy with respect to angles $\vartheta_1,\vartheta_2,\vartheta_3$ must be assessed, giving rise to the following two conditions (both of which must hold in stable configuration)
%\begin{align} \label{eq_3jun_aniso_stabcond1}
%     0 <& \sum_{i=1}^3 \left[ \frac{\mathrm{d}^2 \sigma_i}{\mathrm{d} \vartheta_i^2} + \sigma_i(\vartheta_i) \right]\frac{\sin^2(\vartheta_i)}{d_i} \\  \label{eq_3jun_aniso_stabcond2}
%     0 <& \left[ \frac{\mathrm{d}^2 \sigma_1}{\mathrm{d} \vartheta_1^2} + \sigma_1(\vartheta_1) \right]\left[ \frac{\mathrm{d}^2 \sigma_2}{\mathrm{d} \vartheta_2^2} + \sigma_2(\vartheta_2) \right] \frac{\sin^2(\vartheta_1-\vartheta_2)}{d_1d_2} +    \\
%      &\quad +\left[ \frac{\mathrm{d}^2 \sigma_2}{\mathrm{d} \vartheta_2^2} + \sigma_2(\vartheta_2) \right]\left[ \frac{\mathrm{d}^2 \sigma_3}{\mathrm{d} \vartheta_3^2} +  \sigma_3(\vartheta_3) \right] \frac{\sin^2(\vartheta_2-\vartheta_3)}{d_2d_3} + \nonumber \\
%      &\quad + \left[ \frac{\mathrm{d}^2 \sigma_1}{\mathrm{d} \vartheta_1^2} + \sigma_1(\vartheta_1) \right]\left[ \frac{\mathrm{d}^2 \sigma_3}{\mathrm{d} \vartheta_3^2} + \sigma_3(\vartheta_3) \right]  \frac{\sin^2(\vartheta_3-\vartheta_1)}{d_3d_1} \nonumber \,,
% \end{align}
%where $d_i$ are lengths from the triple junctions to anchor points, which are not relevant for the current work. Each term in the brackets is interface stiffness of the respective interface.
%
%If the second condition is equal to 0, these conditions are not conclusive, i.e. it is not possible to assess the nature of the stationary point (higher-order derivatives are needed then). 
%
%The assessment of stability of triple junction configuration is an important part of discussion of an equilibrium shape of anisotropic particle  on a substrate plane.
%
%\section{Equilibrium shapes of anisotorpic particle on planar substrate}
%This problem has attracted considerable attention since it is of high practical importance, especially in applications like nucleation, preparation of functional deposits for electrochemical catalysis or in solid-state dewetting. The classical solution to the problem is by Kaischew for crystalline anisotropy with faceted chape\textit{(ref 1.50 in Milchev), year 1951} and Winterbottom1967 (interface energy as continuous function). Here it is demonstrated using the capillary vector formalism. 
%
%In the following only 2D cross section is discussed. It is assumed that the particle-substrate interface (denoted here with index 1) is linear with the substrate-parent phase (index 3), but with anti-parallel normal vector. Let the substrate plane be parallel with x axis. The particle-parent phase interface is denoted by index 2. The normal vectors of the respective interfaces in the contact point are $\hat{\bm{n}}_1=(0,-1)^{\mathrm{T}}$, $\hat{\bm{n}}_3=(0,1)^{\mathrm{T}}$ and $\hat{\bm{n}}_2$ is to be found from~\eqref{eq_trijun_forcebalance_sum_xivec}. Apparently, with the interfaces 1 and 3 fixed at orientations $\theta_1=-\pi/2$ and $\theta_3=\pi/2$, the force balance can be reduced to single equation, when the particle-parent phase capillary vector $\bm{\xi}_2$ is projected to the surface normal $\hat{\bm{n}}_S=\hat{\bm{n}}_3$. But that is merely the y-component of the capillary vector. The force balance can thus be written as
%\begin{equation}\label{eq_youngs_eq_aniso}
%     \sigma_{3}(\pi/2)-\sigma_1(-\pi/2)  
%       = \sigma_2(\theta_2)\sin(\theta_2) + \sigma_2'(\theta_2)\cos(\theta_2) \,.
% \end{equation}
%
%Geometric interpretation of this equation is that the force equilibrium in the triple junction is achieved when normal angle $\theta_2$ of the particle-parent phase interface in the triple junction is that one, which has the y coordinate of the Wulff shape equal to the difference $\sigma_{3}(\pi/2)-\sigma_1(-\pi/2) $. The equilibrium shape of anisotropic particle on a substrate is thus that of an isolated particle but truncated at a position, which derives from the force balance in the triple junction. The truncated part of isolated particle counts, which is above the truncating line.
%
%When all the three interfaces are isotropic, the well known Young's equation is obtained
%\begin{align}
%     \sigma_3-\sigma_1 &= \sigma_2\sin(\theta_2) \\
%         &= \sigma_2\cos(\vartheta_2)
%     \,,
% \end{align}
%where in the second equation it was used that $\vartheta_2=\theta_2-\pi/2$.
%
%Let's assume the anisotropy of the particle-parent phase interface $\sigma_2(\theta_2)=\sigma_2^0f(\theta_2)$. Then the wetting parameter $\Gamma$ is defined
%\begin{equation}
%     \Gamma = \frac{\sigma_{3}(\pi/2)-\sigma_1(-\pi/2) }{\sigma_2^0} \,,
% \end{equation}
%which is the vertical offset of the horizontal truncating line relative to center of the equilibrium shape of unit radius. Apparently, when $\Gamma > 0$, the line is above the Wulff shape center (the shape is submerged below the line) and there is such value of $\Gamma$, when the line is merely a tangent. That corresponds to the case of complete wetting and the contact angle is 0. For isotropic interface energy that occurs for $\Gamma=1$.
%
%On the other hand, when $\Gamma < 0$, the truncating line goes below the center (the shape is emerged above the line). There is such value of $\Gamma$ corresponding to complete non-wetting (with isotropic interface energy that is $\Gamma=-1$).
%
%\section{Stability of the equilibrium shape}\label{sec_fund_anisoIE_trijun_stability}
%Previous section showed derivation of the 2D equilibrium shape of anisotropic particle on a substrate from the triple junction force balance. However, as explained in~\ref{sec_intro_trijun_forcebalance_stability}, the stability of the solution should be assessed as well. The additional conditions~\eqref{eq_3jun_aniso_stabcond1}, \eqref{eq_3jun_aniso_stabcond2} can be re-written as
%\begin{align}
%     0 &< \Sigma_1 s_1 + \Sigma_2 s_2 + \Sigma_3 s_3 \\
%     0 &< \Sigma_1\Sigma_2 S_{12} + \Sigma_2\Sigma_3 S_{23} + \Sigma_1\Sigma_3 S_{13} \,,
% \end{align}
%where $\Sigma_i = \mathrm{d}^2 \sigma_i/\mathrm{d} \vartheta_i^2 + \sigma_i(\vartheta_i)$ are the respective interface stiffnesses, $s_i = \sin^2(\vartheta_i)/d_i$, $S_{ij} = \sin^2(\vartheta_i-\vartheta_j)/d_i d_j$. Given $\vartheta_1=0$ and $\vartheta_3=\pi$, it follows that $s_1=s_3=0$ and also that $S_{13}=0$. Note also that $s_2>0$ and $S_{12}=S_{23}=S=\sin^2(\vartheta_2)/d^2>0$ (assuming that $d_1=d_2=d_3$). The conditions are thus simplified to
%\begin{align}
%     0 &< \Sigma_2  \label{eq_3jun_stabcond_onplane1}\\
%     0 &< \Sigma_1\Sigma_2 + \Sigma_2\Sigma_3 \,,  
% \end{align}
%where only the interface stiffnesses matter. Using the first equation, the second may further be simplified
%\begin{equation}
%     0< \Sigma_1 + \Sigma_3 \,. \label{eq_3jun_stabcond_onplane2}
% \end{equation}
%The condition~\eqref{eq_3jun_stabcond_onplane1} implies, that the particle-parent phase interface must be oriented under allowed angle in the contact point. That is fulfilled automatically by the truncated Wulff shape solution, because it contains only the allowed angles. 
%
%The condition~\eqref{eq_3jun_stabcond_onplane2} is surely fulfilled when both interfaces 1 and 3 have positive interface stiffness in their orientations. That holds automatically, when the Wulff shapes of the interfaces 1 and 3 do not contain any corners. When both of them do contain corners, it means that there are normal angles where the interface stiffness is negative. Apparently, in such situation, the condition~~\eqref{eq_3jun_stabcond_onplane2} may be violated.
%
%\section{Summary}
%This chapter contained fundamental information about the equilibrium stable shapes of particles with anisotropic interface energy, assuming absence of other driving forces which might affect their shape. Isolated particles as well as those on a plane were assessed in 2D.

%%%%%%%%%%%%%%%%%%%%%%%%%%%%%%%%%%%%%%%%%%%%%%%%%%
% Keep the following \cleardoublepage at the end of this file, 
% otherwise \includeonly includes empty pages.
\cleardoublepage

% vim: tw=70 nocindent expandtab foldmethod=marker foldmarker={{{}{,}{}}}
